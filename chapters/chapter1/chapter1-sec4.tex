\section{Cosets and Counting}

\setcounter{exercise}{0}

\begin{exercise}
  Let \(G\) be a group and \(\{H_i\mid i\in \}\) a family of subgroups.
  Then for any \(a\in G\), \((\bigcup_i H_i)a=\bigcup_i H_ia\).
\end{exercise}

\begin{solution}
  Let \(G\) be a group, \(\{H_i\mid i\in I\}\) a family of subgroups, and \(a\in G\).
  We prove the set equality
  \[
    \left(\bigcap_{i\in I} H_i\right)a \;=\; \bigcap_{i\in I} (H_i a).
  \]

  \medskip\noindent\textbf{(\(\subseteq\)).}
  Let \(x\in\left(\bigcap_i H_i\right)a\).
  Then \(x=ha\) for some \(h\in\bigcap_i H_i\).
  Thus \(h\in H_i\) for every \(i\), so \(x=ha\in H_i a\) for every \(i\).
  Hence \(x\in\bigcap_i (H_i a)\).

  \medskip\noindent\textbf{(\(\supseteq\)).}
  Let \(x\in\bigcap_i (H_i a)\).
  Then for each \(i\) there exists \(h_i\in H_i\) such that \(x=h_i a\).
  Multiplying on the right by \(a^{-1}\) gives
  \[
    xa^{-1}=h_i\in H_i \quad\text{for all } i,
  \]
  so \(xa^{-1}\in\bigcap_i H_i\).
  Therefore \(x=(xa^{-1})a\in (\bigcap_i H_i)a\).

  \medskip\noindent
  Thus \((\bigcap_i H_i)a=\bigcap_i (H_i a)\).
\end{solution}

\begin{exercise}
  \begin{enumerate}[label=(\alph*)]
    \item Let \(H\) be the cyclic subgroup (of order \(2\)) of \(S_3\) generated by \(\begin{pmatrix}1&2&3\\2&1&3\end{pmatrix}\).
    Then no left coset of \(H\) (except \(H\) itself) is also a right coset.
    There exists \(a\in S_3\) such that \(aH\cap Ha=\{a\}\).
    \item If \(K\) is the cyclic subgroup (of order \(3\)) of \(S_3\) generated by \(\begin{pmatrix}1&2&3\\2&3&1\end{pmatrix}\), then every left coset of \(K\) is also a right coset of \(K\).
  \end{enumerate}
\end{exercise}

\begin{solution}
  Work in \(S_3\).
  Let
  \[
    h=
    \begin{pmatrix}
      1&2&3\\
      2&1&3
    \end{pmatrix}=(12),\qquad H=\langle h\rangle=\{e,(12)\}.
  \]

  \begin{enumerate}[label=(\alph*)]
    \item \textbf{No left coset of \(H\) other than \(H\) is a right coset.}
    The left cosets of \(H\) are \(H\), \((13)H\), and \((23)H\).
    Compute:
    \[
      (13)H=\{(13),(13)(12)\}=\{(13),(123)\},
    \]
    since \((13)(12)=(123)\), and
    \[
      (23)H=\{(23),(23)(12)\}=\{(23),(132)\},
    \]
    since \((23)(12)=(132)\).

    The right cosets are \(H\), \(H(13)\), and \(H(23)\).
    Compute:
    \[
      H(13)=\{(13),(12)(13)\}=\{(13),(132)\},
    \]
    since \((12)(13)=(132)\), and
    \[
      H(23)=\{(23),(12)(23)\}=\{(23),(123)\},
    \]
    since \((12)(23)=(123)\).

    Thus
    \[
      (13)H=\{(13),(123)\}\neq \{(13),(132)\}=H(13),
    \]
    and similarly
    \[
      (23)H=\{(23),(132)\}\neq \{(23),(123)\}=H(23).
    \]
    The remaining left coset is \(H\) itself, which equals the right coset \(H\).
    Hence no left coset of \(H\) (except \(H\)) is also a right coset.

    \smallskip
    \textbf{There exists \(a\in S_3\) with \(aH\cap Ha=\{a\}\).}
    Take \(a=(13)\).
    Then
    \[
      aH=(13)H=\{(13),(123)\},\qquad Ha=H(13)=\{(13),(132)\}.
    \]
    Therefore
    \[
      aH\cap Ha=\{(13)\}=\{a\}.
    \]

    \item Let
    \[
      k=
      \begin{pmatrix}
        1&2&3\\
        2&3&1
      \end{pmatrix}=(123),\qquad K=\langle k\rangle=\{e,(123),(132)\}.
    \]
    This subgroup has index \(2\) in \(S_3\).
    Hence there are exactly two left cosets:
    \(K\) and \(gK\) for any \(g\notin K\); likewise there are exactly two right cosets:
    \(K\) and \(Kg\).

    Take \(g=(12)\notin K\). Then
    \[
      gK=\{(12),(12)(123),(12)(132)\}=\{(12),(23),(13)\},
    \]
    and
    \[
      Kg=\{(12),(123)(12),(132)(12)\}=\{(12),(13),(23)\}.
    \]
    Thus \(gK=Kg\).
    Since any left coset other than \(K\) must equal \(gK\), and any right coset other than \(K\) must equal \(Kg\), it follows that every left coset of \(K\) is also a right coset of \(K\).
  \end{enumerate}
\end{solution}

\begin{exercise}
  The following conditions on a finite group \(G\) are equivalent.
  \begin{enumerate}[label=(\roman*)]
    \item \(G\) is prime.
    \item \(G\ne\langle e\rangle\) and \(G\) has no proper subgroups,
    \item \(G\cong Z_p\) for some prime \(p\).
  \end{enumerate}
\end{exercise}

\begin{solution}
  Let \(G\) be a finite group.
  We prove \((i)\Leftrightarrow(ii)\Leftrightarrow(iii)\).

  \medskip\noindent\textbf{(i)\(\Rightarrow\)(ii).}
  If \(G\) is prime, then \(|G|=p\) for some prime \(p\), so \(G\neq\langle e\rangle\).
  If \(H\le G\) is a subgroup, then \(|H|\) divides \(|G|=p\) (Lagrange's Theorem), hence \(|H|=1\) or \(|H|=p\).
  Therefore \(H=\langle e\rangle\) or \(H=G\), so \(G\) has no proper subgroups.

  \medskip\noindent\textbf{(ii)\(\Rightarrow\)(iii).}
  Assume \(G\neq\langle e\rangle\) and \(G\) has no proper subgroups.
  Pick \(a\in G\) with \(a\neq e\).
  Then \(\langle a\rangle\) is a nontrivial subgroup of \(G\), hence must be all of \(G\).
  Thus \(G\) is cyclic:
  \(G=\langle a\rangle\).
  If \(|G|=n\) were composite, say \(n=rs\) with \(1<r<n\), then by the cyclic-group subgroup theorem, \(G\) would have a (unique) subgroup of order \(r\), which would be proper---contradiction.
  Hence \(|G|=p\) is prime, and \(G\cong Z_p\).

  \medskip\noindent\textbf{(iii)\(\Rightarrow\)(i).}
  If \(G\cong Z_p\) for a prime \(p\), then \(|G|=p\), so \(G\) is prime.

  \medskip\noindent
  Therefore \((i)\), \((ii)\), and \((iii)\) are equivalent.
\end{solution}

\begin{exercise}
  (Euler-Fermat) Let \(a\) be an integer and \(p\) a prime such that \(p\nmid a\).
  Then \(a^{p-1}\equiv1\pmod p\).
  [\emph{Hint}: Consider \(\overline{a}\in Z_p\) and the multiplicative group of nonzero elements of \(Z_p\); see Exercise 1.7.]
  It follows that \(a^p\equiv a\pmod p\) for any integer \(a\).
\end{exercise}

\begin{solution}
  Let \(p\) be prime and let \(a\in\Z\) with \(p\nmid a\).
  Then \(\overline{a}\neq \overline{0}\) in \(Z_p\), so \(\overline{a}\) lies in the multiplicative group
  \[
    Z_p^\times=Z_p\setminus\{\overline{0}\},
  \]
  which has order \(|Z_p^\times|=p-1\) (Exercise~1.7).

  By Lagrange's Theorem, the order of \(\overline{a}\) divides \(|Z_p^\times|=p-1\), hence
  \[
    \overline{a}^{\,p-1}=\overline{1}.
  \]
  Translating back to congruences, this says
  \[
    a^{p-1}\equiv 1 \pmod p.
  \]

  Multiplying both sides by \(a\) gives \(a^p\equiv a\pmod p\) when \(p\nmid a\).
  If \(p\mid a\), then \(a\equiv 0\pmod p\), so \(a^p\equiv 0\equiv a\pmod p\).
  Thus \(a^p\equiv a\pmod p\) for every integer \(a\).
\end{solution}

\begin{exercise}
  Prove that there are only two distinct groups of order \(4\) (up to isomorphism), namely \(Z_4\) and \(Z_2\oplus Z_2\).
  [\emph{Hint}: By Lagrange's Theorem 4.6 a group of order \(4\) that is not cyclic must consist of an identity and three elements of order \(2\).]
\end{exercise}

\begin{solution}
  Let \(G\) be a group with \(|G|=4\).

  \medskip\noindent\textbf{Case 1: \(G\) has an element of order \(4\).}
  Then \(G\) is cyclic, hence \(G\cong Z_4\).

  \medskip\noindent\textbf{Case 2: \(G\) has no element of order \(4\).}
  Then \(G\) is not cyclic.
  By Lagrange's Theorem, the order of any element of \(G\) divides \(4\), so every nonidentity element has order \(2\).
  Thus \(G\) consists of \(e\) and three elements \(a,b,c\) with
  \[
    a^2=b^2=c^2=e.
  \]
  We first show \(G\) is abelian.
  For any \(x,y\in G\), we have
  \[
    (xy)^{-1}=y^{-1}x^{-1}.
  \]
  But every element is its own inverse, so \(x^{-1}=x\) and \(y^{-1}=y\), and also \((xy)^{-1}=xy\).
  Hence
  \[
    xy=(xy)^{-1}=y^{-1}x^{-1}=yx,
  \]
  so \(G\) is abelian.

  Now choose two distinct nonidentity elements, say \(a\neq b\).
  Then \(ab\neq e\) (otherwise \(a=b^{-1}=b\)).
  Also \(ab\neq a\) and \(ab\neq b\) (by cancellation).
  Hence \(ab\) is the third nonidentity element.
  Therefore
  \[
    G=\{e,a,b,ab\}.
  \]
  Define \(\varphi:Z_2\oplus Z_2\to G\) by
  \[
    \varphi(\overline{0},\overline{0})=e,\quad \varphi(\overline{1},\overline{0})=a,\quad \varphi(\overline{0},\overline{1})=b,\quad \varphi(\overline{1},\overline{1})=ab.
  \]
  Since \(a^2=b^2=e\) and \(ab=ba\), one checks that \(\varphi\) is a homomorphism:
  it respects addition mod \(2\) in each coordinate (e.g. \(a\cdot a=e\), \(b\cdot b=e\), and \(a\cdot b=ab\)).
  It is clearly bijective, hence an isomorphism.
  Thus \(G\cong Z_2\oplus Z_2\).

  \medskip\noindent
  Therefore, up to isomorphism, the only groups of order \(4\) are \(Z_4\) and \(Z_2\oplus Z_2\).
\end{solution}

\begin{exercise}
  Let \(H\), \(K\) be subgroups of a group \(G\).
  Then \(HK\) is a subgroup of \(G\) if and only if \(HK=KH\),
\end{exercise}

\begin{solution}
  Let \(H,K<G\).

  \medskip\noindent\textbf{(\(\Rightarrow\))}  
  Assume \(HK\) is a subgroup of \(G\).
  Then \(HK\) is closed under inverses, so
  \[
    (HK)^{-1}=HK.
  \]
  But
  \[
    (HK)^{-1}=\{(hk)^{-1}\mid h\in H,\ k\in K\}=\{k^{-1}h^{-1}\mid h\in H,\ k\in K\}=KH,
  \]
  since \(H\) and \(K\) are subgroups.
  Hence \(HK=KH\).

  \medskip\noindent\textbf{(\(\Leftarrow\))}  
  Conversely, assume \(HK=KH\).
  We show that \(HK\) is a subgroup of \(G\).

  First, \(e=ee\in HK\), so \(HK\neq\varnothing\).
  Let \(x,y\in HK\).
  Then \(x=h_1k_1\) and \(y=h_2k_2\) for some \(h_1,h_2\in H\), \(k_1,k_2\in K\).
  Then
  \[
    xy^{-1}=h_1k_1(k_2^{-1}h_2^{-1})=h_1(k_1k_2^{-1})h_2^{-1}.
  \]
  Since \(k_1k_2^{-1}\in K\) and \(HK=KH\), there exist \(h_3\in H\) and \(k_3\in K\) such that
  \[
    k_1k_2^{-1}h_2^{-1}=h_3k_3.
  \]
  Thus
  \[
    xy^{-1}=h_1h_3k_3\in HK,
  \]
  because \(h_1h_3\in H\) and \(k_3\in K\).
  Hence \(HK\) is closed under \(xy^{-1}\).

  By Theorem~2.5, \(HK\) is a subgroup of \(G\).

  \medskip\noindent
  Therefore \(HK\) is a subgroup of \(G\) if and only if \(HK=KH\).
\end{solution}

\begin{exercise}
  Let \(G\) be a group of order \(p^km\), with \(p\) prime and \((p,m)=1\).
  Let \(H\) be a subgroup of order \(p^k\) and \(K\) a subgroup of order \(p^d\), with \(0<d\le k\) and \(K\not\subset H\).
  Show that \(HK\) is not a subgroup of \(G\).
\end{exercise}

\begin{solution}
  Assume for contradiction that \(HK\) is a subgroup of \(G\).

  Since \(|H|=p^k\) and \(|K|=p^d\), we have \(|H\cap K|=p^r\) for some \(0\le r\le d\).
  Because \(K\not\subset H\), we have \(H\cap K\neq K\), hence \(r<d\).

  Now, since \(HK\) is a subgroup, we have \(HK=KH\) (Exercise~4.6), and thus Theorem~4.7 applies:
  \[
    |HK|=\frac{|H||K|}{|H\cap K|}=\frac{p^k\cdot p^d}{p^r}=p^{k+d-r}.
  \]
  Since \(r<d\), we have \(k+d-r>k\), so \(|HK|\) is a power of \(p\) strictly larger than \(p^k\).

  But \(HK<G\), so by Lagrange's Theorem \(|HK|\) divides \(|G|=p^k m\).
  The only powers of \(p\) dividing \(p^k m\) are at most \(p^k\) (because \((p,m)=1\)).
  Thus no subgroup of \(G\) can have order \(p^{k+d-r}>p^k\), a contradiction.

  Therefore \(HK\) is not a subgroup of \(G\).
\end{solution}

\begin{exercise}
  If \(H\) and \(K\) are subgroups of finite index of a group \(G\) such that \([G:H]\) and \([G:K]\) are relatively prime, then \(G=HK\).
\end{exercise}

\begin{solution}
  Let \(H,K<G\) with \([G:H]=m\), \([G:K]=n\), and \((m,n)=1\).
  Consider \(H\cap K\).

  Since \(H\cap K<H\), we may count cosets inside \(H\):
  \[
    [G:H\cap K]=[G:H]\,[H:H\cap K]=m\,[H:H\cap K].
  \]
  Similarly,
  \[
    [G:H\cap K]=[G:K]\,[K:H\cap K]=n\,[K:H\cap K].
  \]
  Hence \(m\mid [G:H\cap K]\) and \(n\mid [G:H\cap K]\).
  Because \((m,n)=1\), it follows that
  \[
    mn \mid [G:H\cap K].
  \]

  On the other hand, the natural map
  \[
    G/(H\cap K)\;\longrightarrow\; G/H \times G/K,\qquad g(H\cap K)\mapsto (gH,gK)
  \]
  is injective, so
  \[
    [G:H\cap K]\le [G:H]\,[G:K]=mn.
  \]
  Therefore \([G:H\cap K]=mn\).

  Now apply Theorem~4.7 (the product formula) to \(H\) and \(K\):
  \[
    |HK:H|=[K:H\cap K] \quad\text{equivalently}\quad [HK:H]=[K:H\cap K].
  \]
  Translating to indices in \(G\),
  \[
    [G:HK]=\frac{[G:H]}{[HK:H]}=\frac{m}{[K:H\cap K]}.
  \]
  But
  \[
    [K:H\cap K]=\frac{[G:H\cap K]}{[G:K]}=\frac{mn}{n}=m,
  \]
  so \([K:H\cap K]=m\), and hence
  \[
    [G:HK]=\frac{m}{m}=1.
  \]
  Thus \(HK=G\).
\end{solution}

\begin{exercise}
  If \(H\), \(K\) and \(N\) are subgroups of a group \(G\) such that \(H<N\), then \(HK\cap N=H(K\cap N)\).
\end{exercise}

\begin{solution}
  Assume \(H,N,K<G\) and \(H\subset N\).
  We prove
  \[
    HK\cap N \;=\; H(K\cap N).
  \]

  \medskip\noindent\textbf{(\(\subset\)).}
  Let \(x\in HK\cap N\). 
  Then \(x\in HK\), so \(x=hk\) for some \(h\in H\), \(k\in K\).
  Also \(x\in N\).
  Since \(h\in H\subset N\), we have \(h^{-1}\in N\), and therefore
  \[
    k=h^{-1}x \in N.
  \]
  Thus \(k\in K\cap N\), and so \(x=hk\in H(K\cap N)\).

  \medskip\noindent\textbf{(\(\supset\)).}
  Let \(x\in H(K\cap N)\).
  Then \(x=hk\) with \(h\in H\) and \(k\in K\cap N\).
  Clearly \(x\in HK\).
  Also \(h\in H\subset N\) and \(k\in N\), so \(hk\in N\).
  Hence \(x\in HK\cap N\).

  \medskip\noindent
  Therefore \(HK\cap N=H(K\cap N)\).
\end{solution}

\begin{exercise}
  Let \(H\), \(K\), \(N\) be subgroups of a group \(G\) such that \(H<K\), \(H\cap N=K\cap N\), and \(HN=KN\).
  Show that \(H=K\).
\end{exercise}

\begin{solution}
  Assume \(H,K,N<G\) with \(H\subset K\), \(H\cap N=K\cap N\), and \(HN=KN\).
  We prove \(H=K\).

  Since \(H\subset K\), it suffices to show \(K\subset H\).
  Let \(k\in K\).
  Because \(KN=HN\), we have \(k\in KN=HN\), so there exist \(h\in H\) and \(n\in N\) such that
  \[
    k=hn.
  \]
  Then
  \[
    n=h^{-1}k \in K
  \]
  because \(h^{-1}\in H\subset K\) and \(k\in K\).
  Hence \(n\in K\cap N\).
  By the hypothesis \(K\cap N=H\cap N\), it follows that \(n\in H\cap N\subset H\).

  Therefore \(k=hn\in H\), since \(h\in H\) and \(n\in H\).
  Thus \(K\subset H\), and hence \(H=K\).
\end{solution}

\begin{exercise}
  Let \(G\) be a group of order \(2n\); then \(G\) contains an element of order \(2\).
  If \(n\) is odd and \(G\) abelian, there is only one element of order \(2\).
\end{exercise}

\begin{solution}
  Let \(|G|=2n\).

  \medskip\noindent\textbf{Existence of an element of order \(2\).}
  Consider the set \(G-\{e\}\).
  If \(a\in G-\{e\}\) and \(a\neq a^{-1}\), then the elements \(a\) and \(a^{-1}\) form a 2-element pair.
  Thus \(G-\{e\}\) is partitioned into disjoint pairs \(\{a,a^{-1}\}\), together with the elements satisfying \(a=a^{-1}\), i.e. \(a^2=e\).
  If there were no element \(a\neq e\) with \(a^2=e\), then every element of \(G-\{e\}\) would lie in a 2-element pair, so \(|G-\{e\}|\) would be even.
  But
  \[
    |G-\{e\}|=2n-1
  \]
  is odd, a contradiction.
  Hence there exists \(a\neq e\) with \(a^2=e\), i.e. an element of order \(2\).

  \medskip\noindent\textbf{Uniqueness when \(n\) is odd and \(G\) is abelian.}
  Assume now that \(n\) is odd and \(G\) is abelian.
  Let
  \[
    T=\{x\in G\mid x^2=e\}.
  \]
  Then \(T\) is a subgroup of \(G\): it is nonempty, and if \(x^2=e\) and \(y^2=e\), then (using commutativity)
  \[
    (xy)^2=x^2y^2=e,\qquad (x^{-1})^2=(x^2)^{-1}=e,
  \]
  so \(xy\in T\) and \(x^{-1}\in T\).
  Thus \(T<G\).

  Every element of \(T\) has order \(1\) or \(2\), so \(T\) is an elementary abelian \(2\)-group; in particular \(|T|=2^r\) for some \(r\ge 0\).
  By Lagrange's Theorem, \(|T|\) divides \(|G|=2n\).
  Since \(n\) is odd, the largest power of \(2\) dividing \(2n\) is \(2\).
  Hence \(|T|\) must be \(1\) or \(2\).
  But we already proved there exists an element of order \(2\), so \(|T|=2\).

  Therefore \(T=\{e,t\}\) for a unique element \(t\neq e\) with \(t^2=e\), i.e. \(G\) has exactly one element of order \(2\).
\end{solution}

\begin{exercise}
  If \(H\) and \(K\) are subgroups of a group \(G\), then \([H\vee K:H]\ge[K:H\cap K]\).
\end{exercise}

\begin{solution}
  Let \(H,K<G\), and set \(L=H\vee K=\langle H\cup K\rangle\).
  Consider the map
  \[
    \phi:K/(H\cap K)\longrightarrow L/H,\qquad \phi\bigl(k(H\cap K)\bigr)=kH.
  \]
  We first check that \(\phi\) is well defined.
  If \(k(H\cap K)=k'(H\cap K)\), then \(k^{-1}k'\in H\cap K\subset H\), so \(k'H=kH\).
  Hence \(\phi\) is well defined.

  Next we show that \(\phi\) is injective.
  Suppose \(\phi(k(H\cap K))=\phi(k'(H\cap K))\).
  Then \(kH=k'H\), so \(k^{-1}k'\in H\).
  Since also \(k^{-1}k'\in K\), we have \(k^{-1}k'\in H\cap K\), hence \(k(H\cap K)=k'(H\cap K)\).

  Thus \(\phi\) is an injection, so
  \[
    \bigl|K/(H\cap K)\bigr|\ \le\ \bigl|L/H\bigr|.
  \]
  Equivalently,
  \[
    [K:H\cap K]\ \le\ [L:H]=[H\vee K:H].
  \]
  This is the desired inequality.
\end{solution}

\begin{exercise}
  If \(p>q\) are primes, a group of order \(pq\) has at most one subgroup of order \(p\).
  [\emph{Hint}: Suppose \(H\), \(K\) are distinct subgroups of order \(p\).
  Show \(H\cap K=\langle e\rangle\); use Exercise~12 to get a contradiction.]
\end{exercise}

\begin{solution}
  Let \(|G|=pq\) with primes \(p>q\).
  Suppose, for contradiction, that \(G\) has two distinct subgroups \(H\) and \(K\) of order \(p\).

  Since \(|H|=|K|=p\) is prime and \(H\neq K\), we must have
  \[
    H\cap K \neq H \quad\text{and}\quad H\cap K \neq K.
  \]
  By Lagrange's Theorem, \(|H\cap K|\) divides \(|H|=p\), hence \(|H\cap K|=1\) or \(p\).
  The second possibility would force \(H\cap K=H\), i.e. \(H\subset K\), hence \(H=K\), contrary to assumption.
  Therefore
  \[
    H\cap K=\langle e\rangle.
  \]

  Now apply Exercise~12 with these \(H\) and \(K\):
  \[
    [H\vee K:H]\ \ge\ [K:H\cap K]=[K:\langle e\rangle]=|K|=p.
  \]
  Hence
  \[
    |H\vee K| = [H\vee K:H]\cdot |H|\ \ge\ p\cdot p = p^2.
  \]
  But \(H\vee K<G\), so \(|H\vee K|\le |G|=pq\).
  Thus \(p^2\le pq\), which implies \(p\le q\), contradicting \(p>q\).

  Therefore \(G\) has at most one subgroup of order \(p\).
\end{solution}

\begin{exercise}
  Let \(G\) be a group and \(a,b\in G\) such that (i) \(|a| = 4 = |b|\); (ii) \(a^2=b^2\). (iii) \(ba=a^3b=a^{-1}b\); (iv) \(a\ne b\); (v) \(G=\langle a,b\rangle\).
  Show that \(|G|=8\) and \(G\cong Q_8\) (See Exercise 2.3; observe that the generators \(A\), \(B\) of \(Q_8\) also satisfy (i)-(v).)
\end{exercise}

\begin{solution}
  Let \(G\) be a group with elements \(a,b\in G\) satisfying (i)--(v).

  \medskip\noindent\textbf{Step 1: \(a^2=b^2\) is central and has order \(2\).}
  Set \(z=a^2=b^2\).
  Since \(|a|=4\), we have \(z\neq e\) and \(z^2=a^4=e\), so \(|z|=2\).
  Also,
  \[
    az=a(a^2)=a^3=(a^2)a=za,
  \]
  so \(z\) commutes with \(a\).
  And
  \[
    bz=b(b^2)=b^3=(b^2)b=zb,
  \]
  so \(z\) commutes with \(b\).
  Since \(G=\langle a,b\rangle\), it follows that \(z\in Z(G)\).

  \medskip\noindent\textbf{Step 2: Every element of \(G\) can be written as \(a^i b^j\) with \(0\le i\le 3\), \(0\le j\le 1\).}
  From (iii) we have \(ba=a^{-1}b\).
  Multiplying on the right by \(b^{-1}\) gives
  \[
    bab^{-1}=a^{-1}.
  \]
  Equivalently,
  \[
    ba^i = a^{-i}b \qquad\text{for all } i\in\Z,
  \]
  which follows by induction on \(i\) (and also holds for negative \(i\) by inverses).
  Thus any word in \(a^{\pm1}\) and \(b^{\pm1}\) can be rearranged by moving all \(b\)'s to the right at the cost of inverting powers of \(a\), yielding a product \(a^i b^j\).

  Moreover, since \(b^2=z=a^2\), we can reduce any power \(b^j\) to \(j\in\{0,1\}\) at the cost of multiplying by a power of \(a^2\), which is already a power of \(a\).
  And since \(a^4=e\), we can reduce \(i\) modulo \(4\).
  Hence every element of \(G\) is of the form \(a^i b^j\) with \(0\le i\le 3\), \(j\in\{0,1\}\).
  Therefore \(|G|\le 8\).

  \medskip\noindent\textbf{Step 3: These eight elements are distinct, so \(|G|=8\).}
  Consider the set
  \[
    S=\{e,a,a^2,a^3,b,ab,a^2b,a^3b\}.
  \]
  First, \(e,a,a^2,a^3\) are distinct because \(|a|=4\).
  Next, \(b\notin \langle a\rangle\): if \(b=a^i\), then \(b^2=a^{2i}\) would be \(e\) when \(i\) is even or \(a^2\) when \(i\) is odd; but \(b\neq a\) by (iv), and if \(b=a^3\) then \(ba=a^3a=a^4=e\), contradicting (iii).
  Thus \(b\notin\langle a\rangle\).

  Now suppose \(a^i b = a^j b\).
  Right-multiplying by \(b^{-1}\) gives \(a^i=a^j\), so \(i\equiv j\pmod 4\).
  Hence \(b,ab,a^2b,a^3b\) are distinct.
  Also none of \(a^i b\) lies in \(\langle a\rangle\):
  if \(a^i b=a^j\), then \(b=a^{j-i}\in\langle a\rangle\), contradiction.
  Therefore the four elements \(b,ab,a^2b,a^3b\) are distinct from \(e,a,a^2,a^3\).

  Thus \(|S|=8\).
  Since \(G=\langle a,b\rangle\) and every element of \(G\) is of the form \(a^i b^j\), we have \(G=S\).
  Hence \(|G|=8\).

  \medskip\noindent\textbf{Step 4: \(G\cong Q_8\).}
  Let \(Q_8=\langle A,B\rangle\) be the quaternion group, where \(A,B\) satisfy the same relations:
  \[
    |A|=|B|=4,\quad A^2=B^2,\quad BAB^{-1}=A^{-1}.
  \]
  Define \(\varphi:G\to Q_8\) by \(\varphi(a)=A\), \(\varphi(b)=B\).
  By Step 2, every element of \(G\) has a representative \(a^i b^j\) with \(0\le i\le 3\), \(j\in\{0,1\}\).
  Using the relations \(a^4=e\), \(b^2=a^2\), and \(bab^{-1}=a^{-1}\) (and the corresponding relations for \(A,B\)), any two representations of the same element of \(G\) are connected by applications of these relations, so \(\varphi\) is well defined and is a homomorphism.

  Moreover, \(\varphi\) is surjective since \(Q_8=\langle A,B\rangle\).
  Finally, \(|G|=|Q_8|=8\), so a surjective homomorphism \(G\to Q_8\) must be injective as well.
  Therefore \(\varphi\) is an isomorphism, and \(G\cong Q_8\).
\end{solution}
