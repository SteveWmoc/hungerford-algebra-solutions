\section{Cyclic Groups}

\setcounter{exercise}{0}

\begin{exercise}
  Let \(a\), \(b\) be elements of group \(G\).
  Show that \(|a|=|a^{-1}|\); \(|ab|=|ba|\), and \(|a|=|cac^{-1}|\) for all \(c\in G\).
\end{exercise}

\begin{solution}
  Let \(G\) be a group.

  \medskip\noindent\textbf{(1) \(|a|=|a^{-1}|\).}
  If \(|a|=n<\infty\), then \(a^n=e\), hence \((a^{-1})^n=(a^n)^{-1}=e\), so \(|a^{-1}|\mid n\).
  Conversely, if \((a^{-1})^m=e\), then taking inverses gives \(a^m=e\), so \(|a|\mid m\).
  Thus \(|a|=|a^{-1}|\).
  If \(|a|=\infty\) and \((a^{-1})^n=e\) for some \(n\ge1\), then taking inverses gives \(a^n=e\), a contradiction.
  Hence \(|a^{-1}|=\infty\) as well.

  \medskip\noindent\textbf{(2) \(|ab|=|ba|\).}
  Note that
  \[
    ba = a^{-1}(ab)a.
  \]
  Thus \(ba\) is conjugate to \(ab\).
  By part (3) below (applied with \(c=a^{-1}\)), conjugate elements have the same order, so \(|ba|=|ab|\).

  \medskip\noindent\textbf{(3) \(|a|=|cac^{-1}|\) for all \(c\in G\).}
  If \(|a|=n<\infty\), then
  \[
    (cac^{-1})^n = ca^n c^{-1} = cec^{-1}=e,
  \]
  so \(|cac^{-1}|\mid n\).
  Conversely, if \((cac^{-1})^m=e\), then
  \[
    e=(cac^{-1})^m = ca^m c^{-1},
  \]
  so \(a^m=e\), hence \(|a|\mid m\).
  Therefore \(|cac^{-1}|=|a|\).
  If \(|a|=\infty\), the same argument shows \(cac^{-1}\) cannot have finite order, so \(|cac^{-1}|=\infty\).

  Thus \(|a|=|a^{-1}|\), \(|ab|=|ba|\), and \(|a|=|cac^{-1}|\) for all \(c\in G\).
\end{solution}

\begin{exercise}
  Let \(G\) be an abelian group containing elements \(a\) and \(b\) of orders \(m\) and \(n\) respectively.
  Show that \(G\) contains an element whose order is the least common multiple of \(m\) and \(n\).
  [\emph{Hint}: first try the case when \((m,n)=1\).]
\end{exercise}

\begin{solution}
  Let \(G\) be abelian, and let \(|a|=m\), \(|b|=n\).
  Put
  \[
    \ell=\lcm(m,n).
  \]
  We will construct an element of order \(\ell\).

  \medskip\noindent
  \textbf{Lemma.}
  If \(x,y\in G\) commute and \(|x|=r\), \(|y|=s\) with \((r,s)=1\), then \(|xy|=rs\).

  \emph{Proof.}
  Since \(xy=yx\), we have \((xy)^{rs}=x^{rs}y^{rs}=e\), so \(|xy|\mid rs\).
  If \((xy)^k=e\), then \(x^k=y^{-k}\), so \(x^k\in\langle x\rangle\cap\langle y\rangle\).
  Any element of \(\langle x\rangle\cap\langle y\rangle\) has order dividing both \(r\) and \(s\), hence (since \((r,s)=1\)) must be \(e\).
  Thus \(x^k=e=y^k\), so \(r\mid k\) and \(s\mid k\), hence \(rs\mid k\).
  Therefore \(|xy|=rs\). \qed

  \medskip\noindent
  Now write the prime-power factorization
  \[
    \ell=\prod_{p} p^{\gamma_p}, \qquad \gamma_p=\max\{v_p(m),v_p(n)\},
  \]
  where the product is over the finitely many primes dividing \(\ell\).

  For each such prime \(p\), define an element \(x_p\in G\) of order \(p^{\gamma_p}\) as follows.
  If \(\gamma_p=v_p(m)\) (so \(p^{\gamma_p}\mid m\)), set
  \[
    x_p=a^{\,m/p^{\gamma_p}}.
  \]
  Then (in the cyclic subgroup \(\langle a\rangle\)) we have
  \[
    |x_p|=\frac{m}{\gcd\!\left(m,\;m/p^{\gamma_p}\right)}=\frac{m}{m/p^{\gamma_p}}=p^{\gamma_p}.
  \]
  If instead \(\gamma_p=v_p(n)\), set
  \[
    x_p=b^{\,n/p^{\gamma_p}},
  \]
  and the same computation gives \(|x_p|=p^{\gamma_p}\).

  Now define
  \[
  x=\prod_{p\mid \ell} x_p.
  \]
  Since \(G\) is abelian, all the \(x_p\) commute.
  Moreover, the orders \(|x_p|=p^{\gamma_p}\) are pairwise relatively prime for distinct primes \(p\).
  Applying the lemma repeatedly, we obtain
  \[
    |x|=\prod_{p\mid \ell} |x_p|=\prod_{p\mid \ell} p^{\gamma_p}=\ell=\lcm(m,n).
  \]
  Thus \(G\) contains an element of order \(\lcm(m,n)\).
\end{solution}

\begin{exercise}
  Let \(G\) be an abelian group of order \(pq\), with \((p,q)=1\).
  Assume there exist \(a,b\in G\) such that \(|a|=p\), \(|b|=q\) and show that \(G\) is cyclic.
\end{exercise}

\begin{solution}
  Let \(G\) be abelian with \(|G|=pq\), where \((p,q)=1\), and suppose there exist elements \(a,b\in G\) with \(|a|=p\) and \(|b|=q\).

  Since \(G\) is abelian, \(a\) and \(b\) commute.
  By the coprime-order lemma (from the previous exercise), the element
  \[
    x=ab
  \]
  has order
  \[
    |x|=|a||b|=pq.
  \]
  Hence \(\langle x\rangle\) is a cyclic subgroup of \(G\) of order \(pq\).
  But \(|\langle x\rangle|=|G|\), so \(\langle x\rangle=G\).

  Therefore \(G\) is cyclic.
\end{solution}

\begin{exercise}
  If \(f:G\to H\) is a homomorphism, \(a\in G\), and \(f(a)\) has finite order in \(H\), then \(|a|\) is infinite or \(|f(a)|\) divides \(|a|\).
\end{exercise}

\begin{solution}
  Let \(f:G\to H\) be a homomorphism and let \(a\in G\).
  Suppose \(f(a)\) has finite order \(|f(a)|=n\).

  Then \((f(a))^n=e_H\), so
  \[
    e_H=(f(a))^n=f(a^n).
  \]
  Hence \(a^n\in \Ker f\).

  If \(|a|=\infty\), we are done.

  Otherwise \(|a|=m<\infty\).
  Then \(a^m=e_G\), and since \(f(a^n)=e_H\), we have \(f(a)^{\,n}=e_H\).
  By definition of order, \(n\) is the least positive integer with this property.
  But \(f(a)^{\,m}=f(a^m)=e_H\) as well, so \(n\mid m\).
  Therefore \(|f(a)|\) divides \(|a|\).

  Thus \(|a|=\infty\) or \(|f(a)|\mid |a|\).
\end{solution}

\begin{exercise}
  Let \(G\) be the multiplicative group of all nonsingular \(2\times2\) matrices with rational entries.
  Show that \(a=\begin{pmatrix}0&-1\\1&0\end{pmatrix}\) has order \(4\) and \(b=\begin{pmatrix}0&1\\-1&-1\end{pmatrix}\) has order \(3\), but \(ab\) has infinite order.
  Conversely, show that the additive group \(Z_2\oplus\Z\) contains nonzero elements \(a\), \(b\) of infinite order such that \(a+b\) has finite order.
\end{exercise}

\begin{solution}
  Let \(G=GL_2(\Q)\).

  \medskip\noindent\textbf{(1) The elements \(a\) and \(b\).}
  Let
  \[
    a=\begin{pmatrix}0&-1\\[2pt]1&0\end{pmatrix}.
  \]
  Compute
  \[
    a^2=\begin{pmatrix}-1&0\\[2pt]0&-1\end{pmatrix}=-I,\qquad\text{so}\qquad a^4=(a^2)^2=(-I)^2=I.
  \]
  Since \(a\neq I\) and \(a^2=-I\neq I\), it follows that \(|a|=4\).

  Next let
  \[
    b=\begin{pmatrix}0&1\\[2pt]-1&-1\end{pmatrix}.
  \]
  A direct multiplication gives
  \[
  b^2=\begin{pmatrix}-1&-1\\[2pt]1&0\end{pmatrix},\qquad b^3=b^2b=\begin{pmatrix}1&0\\[2pt]0&1\end{pmatrix}=I.
  \]
  Since \(b\neq I\), we conclude \(|b|=3\).

  \medskip\noindent\textbf{(2) The element \(ab\) has infinite order.}
  Compute
  \[
    ab=\begin{pmatrix}0&-1\\[2pt]1&0\end{pmatrix}\begin{pmatrix}0&1\\[2pt]-1&-1\end{pmatrix}=\begin{pmatrix}1&1\\[2pt]0&1\end{pmatrix}=u.
  \]
  We claim \(u^n\neq I\) for all \(n\ge 1\).
  In fact one checks by induction that
  \[
    u^n=\begin{pmatrix}1&n\\[2pt]0&1\end{pmatrix}\qquad(n\in \N^*).
  \]
  Indeed, for \(n=1\) this is \(u\).
  If \(u^n=\begin{pmatrix}1&n\\0&1\end{pmatrix}\), then
  \[
    u^{n+1}=u^n u=\begin{pmatrix}1&n\\[2pt]0&1\end{pmatrix}\begin{pmatrix}1&1\\[2pt]0&1\end{pmatrix}=\begin{pmatrix}1&n+1\\[2pt]0&1\end{pmatrix}.
  \]
  Since \(n\neq 0\) for \(n\ge 1\), we have \(u^n\neq I\).
  Thus \(|ab|=\infty\).

  \medskip\noindent\textbf{(3) In \(Z_2\oplus \Z\) we can have \(|a|=|b|=\infty\) but \(|a+b|<\infty\).}
  Work in the additive group \(Z_2\oplus \Z\).
  Let
  \[
    a=(\overline{1},1),\qquad b=(\overline{1},-1).
  \]
  Then for \(k\in\Z\),
  \[
    k a = (k\overline{1},k) = (\overline{k},k),
  \]
  which equals \((\overline{0},0)\) only when \(k=0\).
  Hence \(|a|=\infty\).
  Similarly \(|b|=\infty\).

  But
  \[
    a+b=(\overline{1}+\overline{1},\,1+(-1))=(\overline{0},0),
  \]
  so \(a+b\) has order \(1\) (finite).

  Thus \(GL_2(\Q)\) contains torsion elements whose product has infinite order, while \(Z_2\oplus\Z\) contains infinite-order elements whose sum has finite order.
\end{solution}

\begin{exercise}
  If \(G\) is a cyclic group of order \(n\) and \(k|n\), then \(G\) has exactly one subgroup of order \(k\).
\end{exercise}

\begin{solution}
  Let \(G=\langle a\rangle\) be a cyclic group of order \(n\), and let \(k\mid n\).

  \medskip\noindent\textbf{Existence.}
  Define
  \[
    H=\langle a^{n/k}\rangle.
  \]
  Since
  \[
    |a^{n/k}|=\frac{n}{\gcd(n,n/k)}=\frac{n}{n/k}=k,
  \]
  the subgroup \(H\) has order \(k\).

  \medskip\noindent\textbf{Uniqueness.}
  Let \(K<G\) be any subgroup of order \(k\).
  Since \(G\) is cyclic, every subgroup of \(G\) is cyclic, so \(K=\langle a^m\rangle\) for some integer \(m\).
  The order of \(a^m\) is
  \[
    |a^m|=\frac{n}{\gcd(n,m)}.
  \]
  Since \(|K|=k\), we must have
  \[
    \frac{n}{\gcd(n,m)}=k \quad\Longrightarrow\quad \gcd(n,m)=\frac{n}{k}.
  \]
  This implies that \(m\) is a multiple of \(n/k\), so
  \[
    \langle a^m\rangle=\langle a^{n/k}\rangle.
  \]
  Hence \(K=H\).

  \medskip\noindent
  Therefore \(G\) has exactly one subgroup of order \(k\).
\end{solution}

\begin{exercise}
  Let \(p\) be prime and \(H\) a subgroup of \(Z(p^\infty)\) (Exercise 1.10).
  \begin{enumerate}[label=(\alph*)]
  \item Every element of \(Z(p^\infty)\) has finite order \(p^n\) for some \(n\geq0\). 
  \item If at least one element of \(H\) has order \(p^k\) and no element of \(H\) has order greater than \(p^k\), then \(H\) is the cyclic subgroup generated by \(\overline{1/p^k}\), whence \(H\cong Z_{p^k}\).
  \item If there is no upper bound on the orders of elements of \(H\), then \(H=Z(p^\infty)\); [see Exercise 2.16].
  \item The only proper subgroups of \(Z(p^\infty)\) are the finite cyclic groups \(C_n=\langle\overline{1/p^n}\rangle\) (\(n=1,2,\dots\)).
  Furthermore, \(\langle0\rangle=C_0<C_1<C_2<C_3<\dots\).
  \item Let \(x_1,x_2,\dots\) be elements of an abelian group \(G\) such that \(|x_1|=p,px_2=x_1,px_3=x_2,\dots,px_{n+1}=x_n,\dots\).
  The subgroup generated by the \(x_i\) (\(i\ge1\)) is isomorphic to \(Z(p^\infty)\).
  [Hint: Verify that the map induced by \(x_i\mapsto\overline{1/p^i}\) is a well-defined isomorphism.]
  \end{enumerate}
\end{exercise}

\begin{solution}
  Fix a prime \(p\).
  Recall
  \[
    Z(p^\infty)=\left\{\overline{a/p^i}\in \Q/\Z\mid a\in\Z,\ i\ge 0\right\},
  \]
  and from Exercise~2.16 it is generated by \(\{\overline{1/p^n}\mid n\ge1\}\).

  \begin{enumerate}[label=(\alph*)]
  \item Let \(x=\overline{a/p^i}\in Z(p^\infty)\).
  Then
  \[
    p^i x=\overline{a}\;=\overline{0},
  \]
  so \(x\) has finite order dividing \(p^i\).
  Hence \(|x|=p^n\) for some \(0\le n\le i\).
  (In particular, \(|\overline{0}|=p^0=1\).)

  \item Assume \(H<\Z(p^\infty)\) contains an element of order \(p^k\) and no element of \(H\) has order \(>p^k\).
  Let \(x\in H\) have order \(p^k\).
  In the cyclic group \(\langle x\rangle\) there is a unique subgroup of order \(p^j\) for each \(0\le j\le k\), and in particular \(\langle x\rangle\) contains an element of order \(p^j\) for each \(j\le k\).

  We claim \(H=\langle x\rangle\).
  Suppose \(y\in H\).
  Then \(|y|=p^t\) for some \(t\le k\) by (a).
  Consider the subgroup \(\langle x\rangle\) of order \(p^k\).
  Since \(Z(p^\infty)\) has exactly one subgroup of order \(p^t\), namely \(\langle \overline{1/p^t}\rangle\) (by the cyclic-group result applied to \(\langle\overline{1/p^k}\rangle\cong Z_{p^k}\)), both \(\langle y\rangle\) and the unique subgroup of \(\langle x\rangle\) of order \(p^t\) must coincide.
  Hence \(y\in \langle x\rangle\).
  Therefore \(H\subseteq\langle x\rangle\), and since \(x\in H\), equality holds: \(H=\langle x\rangle\).

  Finally, any element of order \(p^k\) generates the unique subgroup of order \(p^k\), which is \(\langle\overline{1/p^k}\rangle\).
  Hence
  \[
    H=\left\langle \overline{1/p^k}\right\rangle \cong Z_{p^k}.
  \]

  \item Assume there is no upper bound on the orders of elements of \(H\).
  Then for each \(n\ge1\) there exists \(x_n\in H\) with \(|x_n|\ge p^n\).
  By (a), \(|x_n|\) is a power of \(p\), so in particular \(H\) contains an element of order exactly \(p^n\):
  if \(|x_n|=p^t\) with \(t\ge n\), then \(p^{t-n}x_n\) has order \(p^n\).

  Thus \(H\) contains an element of order \(p^n\) for every \(n\ge1\), hence it contains the unique subgroup of order \(p^n\), namely \(\langle\overline{1/p^n}\rangle\).
  Therefore
  \[
    \left\langle \overline{1/p^n}\right\rangle \subset H \quad\text{for all } n\ge1.
  \]
  But \(Z(p^\infty)\) is generated by \(\{\overline{1/p^n}\mid n\ge1\}\) (Exercise~2.16), so \(H\) contains all generators of \(Z(p^\infty)\), hence \(H=Z(p^\infty)\).

  \item Let \(H\le Z(p^\infty)\) be a proper subgroup.
  By (c), the orders of elements of \(H\) are bounded, so by (b) we have
  \[
    H=\left\langle \overline{1/p^k}\right\rangle =: C_k
  \]
  for some \(k\ge0\) (with \(C_0=\langle 0\rangle\)).
  Hence the only proper subgroups are the finite cyclic groups \(C_n\) (\(n\ge0\)).

  Moreover, since \(\overline{1/p^{n}}\) has order \(p^n\), we have strict inclusions
  \[
    C_0 < C_1 < C_2 < \cdots,
  \]
  and indeed \(C_n\subset C_{n+1}\) because
  \[
    \overline{\frac{1}{p^n}} = p\cdot \overline{\frac{1}{p^{n+1}}}\in C_{n+1}.
  \]

  \item Let \(G\) be abelian and suppose elements \(x_1,x_2,\dots\in G\) satisfy
  \[
    |x_1|=p,\qquad p x_2=x_1,\qquad p x_3=x_2,\ \dots,\ p x_{n+1}=x_n,\ \dots
  \]
  Let \(K=\langle x_i\mid i\ge1\rangle\le G\).
  Define a map on generators by
  \[
    \phi(x_i)=\overline{1/p^i}\in Z(p^\infty).
  \]
  Since \(Z(p^\infty)\) is abelian and \(\phi(px_{i+1})=p\,\phi(x_{i+1})\), we have
  \[
    \phi(px_{i+1}) = p\cdot \overline{1/p^{i+1}}=\overline{1/p^i}=\phi(x_i),
  \]
  so \(\phi\) respects the defining relations \(px_{i+1}=x_i\).
  Also \(|x_1|=p\) matches \(|\overline{1/p}|\! =p\), so no further relation is forced at level \(x_1\).
  Hence \(\phi\) extends to a well-defined homomorphism \(\Phi:K\to Z(p^\infty)\).

  The map \(\Phi\) is surjective because the elements \(\overline{1/p^i}\) generate \(Z(p^\infty)\) (Exercise~2.16), and each \(\overline{1/p^i}\) lies in the image.

  To see injectivity, suppose \(\Phi\bigl(\sum_{i=1}^N c_i x_i\bigr)=0\) for some integers \(c_i\).
  Choose \(N\) large enough so all terms occur.
  Then
  \[
    0=\sum_{i=1}^N c_i\,\overline{1/p^i}\in \mathbb{Q}/\mathbb{Z}.
  \]
  Multiplying by \(p^N\) gives
  \[
    0=\sum_{i=1}^N c_i\, p^{N-i}\,\overline{1}\;=\;\overline{\sum_{i=1}^N c_i p^{N-i}},
  \]
  so \(\sum_{i=1}^N c_i p^{N-i}\in \Z\).
  But this holds automatically; what we really get is that \(\sum_{i=1}^N c_i/p^i\in\Z\), hence \(\sum_{i=1}^N c_i/p^i=0\) in \(Z(p^\infty)\).
  Using the relations \(x_i=p^{N-i}x_N\), we have in \(K\):
  \[
    \sum_{i=1}^N c_i x_i = \left(\sum_{i=1}^N c_i p^{N-i}\right) x_N.
  \]
  The coefficient is divisible by \(p^N\) exactly when \(\sum c_i/p^i\in\Z\), so the above element is \(0\) in \(K\) because \(p^N x_N = x_0:=0\).
  Therefore \(\ker\Phi=0\), and \(\Phi\) is injective.

  Hence \(\Phi\) is an isomorphism \(K\cong Z(p^\infty)\).
\end{enumerate}
\end{solution}

\begin{exercise}
  A group that has only a finite number of subgroups must be finite.
\end{exercise}

\begin{solution}
  Assume \(G\) is a group with only finitely many subgroups.
  We prove \(G\) is finite by contrapositive.

  Suppose \(G\) is infinite.
  Choose an element \(a\in G\) with \(a\neq e\).
  If \(a\) has infinite order, then \(G\) contains the infinite cyclic subgroup \(\langle a\rangle\cong \Z\).
  But \(\Z\) has infinitely many distinct subgroups \(n\Z\) (\(n\in\N^*\)), hence \(\langle a\rangle\) has infinitely many subgroups, and therefore \(G\) has infinitely many subgroups.

  If instead every nonidentity element of \(G\) has finite order, then \(G\) is an infinite torsion group.
  Pick an infinite sequence of distinct elements \(a_1,a_2,\dots\) in \(G\).
  The cyclic subgroups \(\langle a_i\rangle\) are finite.
  If only finitely many distinct cyclic subgroups occurred among them, then their union would be a finite union of finite sets, hence finite, contradicting that \(\{a_i\}\) is infinite.
  Therefore the subgroups \(\langle a_i\rangle\) yield infinitely many distinct subgroups of \(G\).

  In either case, an infinite group has infinitely many subgroups.
  Hence, if \(G\) has only finitely many subgroups, \(G\) must be finite.
\end{solution}

\begin{exercise}
  If \(G\) is an abelian group, then the set \(T\) of all elements of \(G\) with finite order is a subgroup of \(G\). [Compare Exercise 5.]
\end{exercise}

\begin{solution}
  Let \(G\) be an abelian group and let
  \[
    T=\{x\in G \mid |x|<\infty\}.
  \]
  We show \(T<G\) using Theorem~2.5.

  First, \(e\in T\), since \(|e|=1\), so \(T\neq\varnothing\).
  Let \(a,b\in T\).
  Then \(|a|=m\) and \(|b|=n\) for some positive integers \(m,n\).
  Because \(G\) is abelian, \(ab^{-1}=ab^{-1}=a(b^{-1})\) and \(a\) commutes with \(b^{-1}\).
  Also \(|b^{-1}|=|b|=n\).
  Hence
  \[
    (ab^{-1})^{mn}=a^{mn}(b^{-1})^{mn}=(a^m)^n\bigl((b^{-1})^n\bigr)^m=e,
  \]
  so \(ab^{-1}\) has finite order, i.e. \(ab^{-1}\in T\).

  Therefore \(T\) is nonempty and closed under \(ab^{-1}\); by Theorem~2.5, \(T\) is a subgroup of \(G\).
\end{solution}

\begin{exercise}
  An infinite group is cyclic if and only if it is isomorphic to each of its proper subgroups.
\end{exercise}

\begin{solution}
  \textbf{(\(\Rightarrow\))} Suppose \(G\) is infinite cyclic.
  Then \(G\cong \Z\).
  Every proper subgroup of \(\Z\) is of the form \(n\Z\) for some integer \(n\ge 2\), and the map
  \[
    \Z\to n\Z,\qquad k\mapsto nk
  \]
  is an isomorphism (additively).
  Hence every proper subgroup of \(G\) is isomorphic to \(G\).

  \medskip\noindent\textbf{(\(\Leftarrow\))} Conversely, suppose \(G\) is an infinite group that is isomorphic to each of its proper subgroups.

  First, \(G\) must contain an element of infinite order.
  Indeed, if every element of \(G\) had finite order, then every cyclic subgroup \(\langle x\rangle\) would be finite.
  Choose any \(x\neq e\); then \(\langle x\rangle\) is a proper finite subgroup, so \(G\cong \langle x\rangle\) would force \(G\) to be finite, a contradiction.
  Thus there exists \(a\in G\) with \(|a|=\infty\).

  Now consider the cyclic subgroup \(\langle a\rangle\).
  It is infinite, hence \(\langle a\rangle\cong\Z\). If \(\langle a\rangle=G\), then \(G\) is cyclic and we are done.
  If \(\langle a\rangle\ne G\), then \(\langle a\rangle\) is a proper subgroup, so by hypothesis \(G\cong \langle a\rangle\).
  Therefore \(G\cong \Z\), and in particular \(G\) is cyclic.

  Hence an infinite group is cyclic if and only if it is isomorphic to each of its proper subgroups.
\end{solution}

