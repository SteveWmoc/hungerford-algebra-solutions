\section{Homomorphisms and Subgroups}

\setcounter{exercise}{0}

\begin{exercise}
  If \(f:G\to H\) is a homomorphism of groups, then \(f(e_G)=e_H\) and \(f(a^{-1})=f(a)^{-1}\) for all \(a\in G\).
  Show by example that the first conclusion may be false if \(G\), \(H\) are monoids that are not groups.
\end{exercise}

\begin{solution}
  Let \(f:G\to H\) be a group homomorphism.

  \medskip\noindent\textbf{(1) \(f(e_G)=e_H\).}
  Since \(e_Ge_G=e_G\), applying \(f\) and using the homomorphism property gives
  \[
    f(e_G)=f(e_Ge_G)=f(e_G)f(e_G).
  \]
  Multiply on the left by \(f(e_G)^{-1}\) (which exists because \(H\) is a group) to obtain \(e_H=f(e_G)\).
  Hence \(f(e_G)=e_H\).

  \medskip\noindent\textbf{(2) \(f(a^{-1})=f(a)^{-1}\) for all \(a\in G\).}
  We have \(aa^{-1}=e_G\).
  Applying \(f\) gives
  \[
    f(a)f(a^{-1})=f(aa^{-1})=f(e_G)=e_H.
  \]
  Thus \(f(a^{-1})\) is an inverse of \(f(a)\), so \(f(a^{-1})=f(a)^{-1}\).

  \medskip\noindent\textbf{Monoid counterexample.}
  The conclusion \(f(e_G)=e_H\) can fail for homomorphisms of monoids that are not groups, because cancellation/inverses need not exist in the codomain.

  Let \(G=(\mathbb{N},\cdot)\) with identity \(1\), and let \(H=(\mathbb{N},\cdot)\) with identity \(1\).
  Define \(f(n)=0\) for all \(n\).
  Then \(f(mn)=0=0\cdot 0=f(m)f(n)\), so \(f\) is a monoid homomorphism, but
  \[
    f(e_G)=f(1)=0\neq 1=e_H.
  \]
  Thus \(f(e_G)=e_H\) may fail for monoids that are not groups.
\end{solution}

\begin{exercise}
  A group \(G\) is abelian if and only if the map \(G\to G\) given by \(x\mapsto x^{-1}\) is an automorphism.
\end{exercise}

\begin{solution}
  Define \(\iota:G\to G\) by \(\iota(x)=x^{-1}\).

  \medskip\noindent\textbf{(\(\Rightarrow\))}
  If \(G\) is abelian, then for all \(x,y\in G\),
  \[
    \iota(xy)=(xy)^{-1}=y^{-1}x^{-1}=x^{-1}y^{-1}=\iota(x)\iota(y),
  \]
  so \(\iota\) is a homomorphism.
  Since \(\iota\circ\iota=\mathrm{id}_G\), it is bijective.
  Hence \(\iota\) is an automorphism.

  \medskip\noindent\textbf{(\(\Leftarrow\))}
  If \(\iota\) is an automorphism, then it is a homomorphism, so
  \[
    (xy)^{-1}=\iota(xy)=\iota(x)\iota(y)=x^{-1}y^{-1}\qquad(\forall x,y\in G).
  \]
  By Exercise~11 of \S1.1 (equivalent conditions for a group to be abelian), the identity \((xy)^{-1}=x^{-1}y^{-1}\) for all \(x,y\) implies that \(G\) is abelian.

  Therefore \(G\) is abelian if and only if \(x\mapsto x^{-1}\) is an automorphism.
\end{solution}

\begin{exercise}
  Let \(Q_8\) be the group (under ordinary matrix multiplication) generated by the complex matrices \(A=\begin{pmatrix}0&1\\-1&0\end{pmatrix}\) and \(B=\begin{pmatrix}0&i\\i&0\end{pmatrix}\), where \(i^2=-1\).
  Show that \(Q_8\) is a nonabelian group of order \(8\).
  \(Q_8\) is called the \textbf{quaternion group}.
  [\emph{Hint}: Observe that \(BA=A^3B\), whence every element of \(Q_8\) is of the form \(A^iB^j\) Note also that \(A^4=B^4=I\), where \(I=\begin{pmatrix}1&0\\0&1\end{pmatrix}\) is the identity element of \(Q_8\).]
\end{exercise}

\begin{solution}
  Let
  \[
  A=\begin{pmatrix}0&1\\-1&0\end{pmatrix},\qquad B=\begin{pmatrix}0&i\\ i&0\end{pmatrix}\qquad(i^2=-1),
  \]
  and let \(Q_8=\langle A,B\rangle\le GL_2(\mathbb{C})\).

  \medskip\noindent\textbf{Step 1: Use Theorem 2.8 to describe elements of \(\langle A,B\rangle\).}
  By Theorem~2.8, the subgroup \(\langle A,B\rangle\) consists of all finite products in which the factors are powers of \(A\) and \(B\), i.e. every element of \(Q_8\) can be written as a word of the form
  \[
    A^{m_1}B^{n_1}A^{m_2}B^{n_2}\dotsb A^{m_t}B^{n_t},\qquad (m_k,n_k\in\mathbb{Z}).
  \]

  \medskip\noindent\textbf{Step 2: Basic relations.}
  Direct computation gives
  \[
    A^2=\begin{pmatrix}-1&0\\0&-1\end{pmatrix}=-I \quad\Rightarrow\quad A^4=I,
  \]
  \[
    B^2=\begin{pmatrix}-1&0\\0&-1\end{pmatrix}=-I \quad\Rightarrow\quad B^4=I.
  \]
  Also
  \[
    AB=\begin{pmatrix}i&0\\0&-i\end{pmatrix},\qquad BA=\begin{pmatrix}-i&0\\0&i\end{pmatrix}=-AB.
  \]
  Since \(A^3=-A\), we have \(A^3B=-(AB)=BA\); hence
  \begin{equation}\label{eq:BA}
    BA=A^3B.
  \end{equation}

  \medskip\noindent\textbf{Step 3: Normal form \(A^iB^j\) with \(0\le i\le 3\), \(j\in\{0,1\}\).}
  Using \eqref{eq:BA}, we can move any \(B\) past any \(A\) to the right at the cost of replacing \(A\) by \(A^3=A^{-1}\).
  Repeating this process, any word
  \[
    A^{m_1}B^{n_1}\dotsb A^{m_t}B^{n_t}
  \]
  can be rewritten as \(A^iB^j\) for some integers \(i,j\).
  Reducing exponents modulo \(4\) using \(A^4=B^4=I\), we may assume \(0\le i\le 3\) and \(0\le j\le 3\).
  But since \(B^2=-I=A^2\), we have
  \[
    A^iB^j =
    \begin{cases}
      A^i,& j\equiv 0\pmod 2,\\
      A^iB,& j\equiv 1\pmod 2,
    \end{cases}
  \]
  so in fact every element is of the form \(A^iB^j\) with \(0\le i\le 3\) and \(j\in\{0,1\}\).
  Hence \(|Q_8|\le 8\).

  \medskip\noindent\textbf{Step 4: There are at least eight distinct elements.}
  The eight matrices
  \[
    I,\ A,\ A^2=-I,\ A^3=-A,\ B,\ AB,\ A^2B=-B,\ A^3B=-AB
  \]
  are all distinct.
  Indeed, the first four have only real entries, whereas \(B,AB,-B,-AB\) have nonreal entries, so no \(A^i\) can equal any \(A^kB\).
  Also \(B\neq -B\), \(AB\neq -AB\), and \(B\neq \pm AB\) since \(B\) is off-diagonal while \(AB\) is diagonal.
  Thus \(|Q_8|\ge 8\).

  Combining with \(|Q_8|\le 8\), we conclude \(|Q_8|=8\), and
  \[
    Q_8=\{\pm I,\ \pm A,\ \pm B,\ \pm AB\}.
  \]

  \medskip\noindent\textbf{Step 5: Nonabelian.}
  Since \(BA=-AB\) and \(AB\neq BA\), the group \(Q_8\) is not abelian.

  Therefore \(Q_8\) is a nonabelian group of order \(8\).
\end{solution}

\begin{exercise}
  Let \(H\) be the group (under matrix multiplication) of real matrices generated by \(C=\begin{pmatrix}0&1\\-1&0\end{pmatrix}\) and \(D=\begin{pmatrix}0&1\\1&0\end{pmatrix}\).
  Show that \(H\)is a nonabelian group of order \(8\) which is not isomorphic to the quaternion group of Exercise 3, but is isomorphic to the group \(D_4^{\phantom{4}*}\).
\end{exercise}

\begin{solution}
  Let
  \[
    C=\begin{pmatrix}0&1\\-1&0\end{pmatrix},\qquad D=\begin{pmatrix}0&1\\1&0\end{pmatrix},
  \]
  and let \(H=\langle C,D\rangle\le GL_2(\mathbb{R})\).

  \medskip\noindent\textbf{Step 1: Relations and a normal form.}
  A direct computation gives
  \[
    C^2=\begin{pmatrix}-1&0\\0&-1\end{pmatrix}=-I \quad\Rightarrow\quad C^4=I,\qquad D^2=I.
  \]
  Also
  \[
    DC=\begin{pmatrix}-1&0\\0&1\end{pmatrix},\qquad CD=\begin{pmatrix}1&0\\0&-1\end{pmatrix},
  \]
  so \(DC=-\,CD\).
  Since \(C^3=-C\), we have
  \[
    C^{-1}=C^3=-C,
  \]
  and hence
  \[
    DCD = -C = C^{-1}.
  \]
  Equivalently,
  \begin{equation}\label{eq:dihedral-relation}
    DC=C^{-1}D.
  \end{equation}
  Using \eqref{eq:dihedral-relation}, any word in \(C\) and \(D\) can be rewritten by moving each \(D\) to the right at the cost of inverting a power of \(C\).
  By Theorem~2.8, every element of \(H\) is a finite product of powers of \(C\) and \(D\), hence every element can be written in the form \(C^iD^j\) with \(i\in\mathbb{Z}\), \(j\in\{0,1\}\).
  Using \(C^4=I\), we may assume \(0\le i\le 3\).
  Therefore
  \[
    H\subset \{\,C^iD^j : 0\le i\le 3,\ j\in\{0,1\}\,\},
  \]
  so \(|H|\le 8\).

  \medskip\noindent\textbf{Step 2: There are eight distinct elements and \(H\) is nonabelian.}
  The matrices
  \[
    I,\ C,\ C^2=-I,\ C^3=-C,\ D,\ CD,\ C^2D=-D,\ C^3D=-CD
  \]
  are all distinct (for instance, \(D\) is symmetric while \(CD\) is diagonal).
  Hence \(|H|\ge 8\), so \(|H|=8\).
  Moreover \(CD\neq DC\) (indeed \(DC=-CD\)), so \(H\) is nonabelian.

  \medskip\noindent\textbf{Step 3: \(H\) is not isomorphic to \(Q_8\).}
  In \(H\), the element \(D\neq I\) satisfies \(D^2=I\), so \(H\) has an element of order \(2\).
  In the quaternion group \(Q_8=\{\pm I,\pm A,\pm B,\pm AB\}\), the only element of order \(2\) is \(-I\); all other nonidentity elements have order \(4\).
  Therefore \(H\nsim Q_8\), since an isomorphism preserves element orders.

  \medskip\noindent\textbf{Step 4: \(H\sim D_4^{\phantom{4}*}\).}
  Let \(D_4^{\phantom{4}*}=\langle r,s\mid r^4=e,\ s^2=e,\ srs=r^{-1}\rangle\) be the dihedral group of order \(8\).
  Define a map \(\varphi:D_4^{\phantom{4}*}\to H\) on generators by
  \[
    \varphi(r)=C,\qquad \varphi(s)=D.
  \]
  The defining relations are satisfied in \(H\):
  \[
    \varphi(r)^4=C^4=I,\qquad \varphi(s)^2=D^2=I,\qquad\varphi(s)\varphi(r)\varphi(s)=DCD=C^{-1}=\varphi(r)^{-1}.
  \]
  Hence \(\varphi\) extends to a well-defined homomorphism \(D_4^{\phantom{4}*}\to H\).
  Its image contains \(C\) and \(D\), so it contains \(\langle C,D\rangle=H\); thus \(\varphi\) is surjective.
  Since \(|D_4^{\phantom{4}*}|=8=|H|\), a surjective homomorphism between finite groups of the same order is injective.
  Therefore \(\varphi\) is an isomorphism, and \(H\sim D_4^{\phantom{4}*}\).

  Thus \(H\) is a nonabelian group of order \(8\), not isomorphic to \(Q_8\), but isomorphic to \(D_4^{\phantom{4}*}\).
\end{solution}

\begin{exercise}
  Let \(S\) be a nonempty subset of a group \(G\) and define a relation on \(G\) by \(a\sim b\) if and only if \(ab^{-1}\in S\).
  Show that \(\sim\) is an equivalence relation if and only if \(S\) is a subgroup of \(G\).
\end{exercise}

\begin{solution}
  Let \(S\) be a nonempty subset of a group \(G\), and define a relation on \(G\) by
  \[
    a\sim b \iff ab^{-1}\in S.
  \]

  \medskip\noindent\textbf{(\(\Rightarrow\))}
  Assume \(\sim\) is an equivalence relation.
  We show that \(S\) is a subgroup of \(G\).

  Since \(\sim\) is reflexive, for every \(a\in G\) we have \(a\sim a\), hence \(aa^{-1}=e\in S\).
  In particular, \(S\) is nonempty and contains the identity.

  Let \(x,y\in S\).
  Because \(x\in S\), we have \(x\sim e\); because \(y\in S\), we have \(y\sim e\).
  Since \(\sim\) is symmetric, \(e\sim y\), and since it is transitive,
  \[
    x\sim e \ \text{and}\ e\sim y \ \Longrightarrow\ x\sim y.
  \]
  Thus \(xy^{-1}\in S\).

  Therefore \(S\) is nonempty and satisfies \(xy^{-1}\in S\) for all \(x,y\in S\).
  By Theorem~2.5, \(S\) is a subgroup of \(G\).

  \medskip\noindent\textbf{(\(\Leftarrow\))}
  Conversely, assume \(S\) is a subgroup of \(G\).
  We verify that \(\sim\) is an equivalence relation.

  \begin{itemize}
    \item \emph{Reflexive:} For any \(a\in G\), \(aa^{-1}=e\in S\), so \(a\sim a\).
    \item \emph{Symmetric:} If \(a\sim b\), then \(ab^{-1}\in S\).
    Since \(S\) is a subgroup, \((ab^{-1})^{-1}=ba^{-1}\in S\), so \(b\sim a\).
    \item \emph{Transitive:} If \(a\sim b\) and \(b\sim c\), then \(ab^{-1}\in S\) and \(bc^{-1}\in S\).
    Since \(S\) is a subgroup,
    \[
      (ab^{-1})(bc^{-1})=ac^{-1}\in S,
    \]
    so \(a\sim c\).
  \end{itemize}

  Thus \(\sim\) is an equivalence relation.

  \medskip\noindent
  Hence \(\sim\) is an equivalence relation on \(G\) if and only if \(S\) is a subgroup of \(G\).
\end{solution}

\begin{exercise}
  A nonempty finite subset of a group is a subgroup if and only if it is closed under the product in \(G\).
\end{exercise}

\begin{solution}
  Let \(H\) be a nonempty finite subset of a group \(G\).

  \medskip\noindent\textbf{(\(\Rightarrow\))}
  If \(H\) is a subgroup of \(G\), then it is closed under the product in \(G\) by definition.

  \medskip\noindent\textbf{(\(\Leftarrow\))} Conversely, assume \(H\) is closed under the product in \(G\).
  We show that \(H\) is a subgroup.

  Fix \(a\in H\).
  Consider the map \(L_a:H\to H\) given by \(L_a(x)=ax\).
  Closure under products implies \(L_a\) is well defined.
  Moreover, \(L_a\) is injective:
  if \(ax=ay\), then by left cancellation in \(G\) we have \(x=y\).
  Since \(H\) is finite, \(L_a\) is surjective.
  Therefore there exists \(e\in H\) such that \(L_a(e)=ae=a\).
  Cancelling \(a\) on the left gives \(e=e_G\), so \(e_G\in H\).

  Next, since \(L_a\) is surjective and \(e_G\in H\), there exists \(b\in H\) such that \(ab=e_G\).
  Then \(b=a^{-1}\).
  Hence \(a^{-1}\in H\) for every \(a\in H\).

  Now \(H\) is nonempty, closed under products, contains \(e_G\), and contains inverses; therefore \(H\) is a subgroup of \(G\).

  (Equivalently, one may apply Theorem~2.5: since \(a^{-1}\in H\), we have \(ab^{-1}\in H\) for all \(a,b\in H\), so \(H\le G\).)
\end{solution}

\begin{exercise}
  If \(n\) is a fixed integer, then \(\{kn\mid k\in\Z\}\subset\Z\) is an additive subgroup of \(\Z\), which is isomorphic to \(\Z\).
\end{exercise}

\begin{solution}
  Fix an integer \(n\) and let
  \[
    n\mathbb{Z}=\{kn\mid k\in\mathbb{Z}\}\subset \mathbb{Z}.
  \]

  \medskip\noindent\textbf{Subgroup.}
  The set \(n\mathbb{Z}\) is nonempty since \(0=0\cdot n\in n\mathbb{Z}\).
  If \(kn,\ell n\in n\mathbb{Z}\), then
  \[
  kn+\ell n=(k+\ell)n\in n\mathbb{Z}.
  \]
  Also \(-(kn)=(-k)n\in n\mathbb{Z}\).
  Hence \(n\mathbb{Z}\) is a subgroup of the additive group \((\mathbb{Z},+)\).

  \medskip\noindent\textbf{Isomorphism with \(\mathbb{Z}\) (for \(n\neq 0\)).}
  Assume \(n\neq 0\).
  Define \(\varphi:\mathbb{Z}\to n\mathbb{Z}\) by
  \[
    \varphi(k)=kn.
  \]
  Then \(\varphi\) is a homomorphism:
  \[
    \varphi(k+\ell)=(k+\ell)n=kn+\ell n=\varphi(k)+\varphi(\ell).
  \]
  It is surjective by definition of \(n\mathbb{Z}\). If \(\varphi(k)=\varphi(\ell)\), then \(kn=\ell n\), so \((k-\ell)n=0\).
  Since \(n\neq 0\), it follows that \(k-\ell=0\), hence \(k=\ell\).
  Thus \(\varphi\) is injective.
  Therefore \(\varphi\) is an isomorphism, and \(n\mathbb{Z}\cong \mathbb{Z}\).

  \medskip\noindent\textbf{Remark (the case \(n=0\)).}
  If \(n=0\), then \(n\mathbb{Z}=\{0\}\), the trivial subgroup, which is not isomorphic to \(\mathbb{Z}\).
\end{solution}

\begin{exercise}
  The set \(\{\sigma\in S_n\mid\sigma(n)=n\}\) is a subgroup of \(S_n\) which is isomorphic to \(S_{n-1}\).
\end{exercise}

\begin{solution}
  Let
  \[
    H=\{\sigma\in S_n \mid \sigma(n)=n\}.
  \]

  \medskip\noindent\textbf{\(H\) is a subgroup of \(S_n\).}
  
  The identity permutation satisfies \(e(n)=n\), so \(e\in H\), hence \(H\neq\varnothing\).
  If \(\sigma,\tau\in H\), then
  \[
    (\sigma\tau)(n)=\sigma(\tau(n))=\sigma(n)=n,
  \]
  so \(\sigma\tau\in H\).
  If \(\sigma\in H\), then \(\sigma(n)=n\) implies \(\sigma^{-1}(n)=n\) (apply \(\sigma^{-1}\) to both sides), so \(\sigma^{-1}\in H\).
  Thus \(H< S_n\).

  \medskip\noindent\textbf{\(H\sim S_{n-1}\).}
  
  Define a map
  \[
    \varphi:H\to S_{n-1}
  \]
  by restriction:
  for \(\sigma\in H\), let \(\varphi(\sigma)\) be the permutation of \(\{1,2,\dots,n-1\}\) given by \(\varphi(\sigma)(k)=\sigma(k)\).
  This is well defined:
  since \(\sigma(n)=n\), the set \(\{1,\dots,n-1\}\) is \(\sigma\)-invariant, so \(\sigma(k)\in\{1,\dots,n-1\}\) whenever \(k\le n-1\).

  Moreover \(\varphi\) is a homomorphism because restriction commutes with composition:
  \[
    \varphi(\sigma\tau)(k)=(\sigma\tau)(k)=\sigma(\tau(k))=\varphi(\sigma)(\varphi(\tau)(k))=(\varphi(\sigma)\varphi(\tau))(k).
  \]

  It is injective:
  if \(\varphi(\sigma)=\varphi(\tau)\), then \(\sigma(k)=\tau(k)\) for all \(k\le n-1\), and also \(\sigma(n)=n=\tau(n)\), hence \(\sigma=\tau\).

  It is surjective:
  given any \(\pi\in S_{n-1}\), define \(\widetilde{\pi}\in S_n\) by
  \[
    \widetilde{\pi}(k)=\pi(k)\ (1\le k\le n-1),\qquad \widetilde{\pi}(n)=n.
  \]
  Then \(\widetilde{\pi}\in H\) and \(\varphi(\widetilde{\pi})=\pi\).

  Thus \(\varphi\) is a bijective homomorphism, so \(H\sim S_{n-1}\).
\end{solution}

\begin{exercise}
  Let \(f:G\to H\) be a homomorphism of groups, \(A\) a subgroup of \(G\), and \(B\) a subgroup of \(H\). 
  \begin{enumerate}[label=(\alph*)]
    \item \(\Ker f\) and \(f^{-1}(B)\) are subgroups of \(G\).
    \item \(f(A)\) is a subgroup of \(H\).
  \end{enumerate}
\end{exercise}

\begin{solution}
  Let \(f:G\to H\) be a group homomorphism, \(A<G\), and \(B<H\).

  \begin{enumerate}[label=(\alph*)]
    \item \textbf{\(\Ker f\) and \(f^{-1}(B)\) are subgroups of \(G\).}

    Recall \(\Ker f=\{g\in G: f(g)=e_H\}\).
    It is nonempty since \(f(e_G)=e_H\), so \(e_G\in\Ker f\).
    If \(x,y\in\Ker f\), then
    \[
      f(xy^{-1})=f(x)f(y^{-1})=f(x)\,f(y)^{-1}=e_H e_H^{-1}=e_H,
    \]
    so \(xy^{-1}\in\Ker f\).
    By Theorem~2.5, \(\Ker f<G\).

    Next, \(f^{-1}(B)=\{g\in G: f(g)\in B\}\) is nonempty since \(e_H\in B\) and \(e_G\in f^{-1}(B)\).
    If \(x,y\in f^{-1}(B)\), then \(f(x),f(y)\in B\), and since \(B\le H\),
    \[
      f(xy^{-1})=f(x)f(y)^{-1}\in B.
    \]
    Hence \(xy^{-1}\in f^{-1}(B)\).
    By Theorem~2.5, \(f^{-1}(B)<G\).

    \item \textbf{\(f(A)\) is a subgroup of \(H\).}

    First, \(f(A)\neq\varnothing\) since \(e_G\in A\) implies \(e_H=f(e_G)\in f(A)\).
    Let \(u,v\in f(A)\).
    Then \(u=f(a)\) and \(v=f(b)\) for some \(a,b\in A\).
    Since \(A\le G\), we have \(ab^{-1}\in A\).
    Therefore
    \[
      uv^{-1}=f(a)\,f(b)^{-1}=f(a)\,f(b^{-1})=f(ab^{-1})\in f(A).
    \]
    By Theorem~2.5 (applied in \(H\)), it follows that \(f(A)<H\).
  \end{enumerate}
\end{solution}

\begin{exercise}
  List all subgroups of \(Z_2\oplus Z_2\).
  Is \(Z_2\oplus Z_2\) isomorphic to \(Z_4\)?
\end{exercise}

\begin{solution}
  Write \(V=Z_2\oplus Z_2=\{(0,0),(1,0),(0,1),(1,1)\}\) under componentwise addition mod \(2\).

  \medskip\noindent\textbf{Subgroups.}
  Every subgroup of \(V\) must contain \((0,0)\).
  Also, since every nonidentity element has order \(2\), any subgroup generated by a nonzero element has exactly two elements.

  Thus the subgroups are:
  \[
    \{(0,0)\},
  \]
  \[
    \langle(1,0)\rangle=\{(0,0),(1,0)\},\qquad\langle(0,1)\rangle=\{(0,0),(0,1)\},\qquad\langle(1,1)\rangle=\{(0,0),(1,1)\},
  \]
  and the whole group
  \[
    V=\{(0,0),(1,0),(0,1),(1,1)\}.
  \]
  There are no other subgroups:
  any subgroup containing two distinct nonzero elements contains their sum as well, hence all three nonzero elements, and so it must be all of \(V\).

  \medskip\noindent\textbf{Is \(Z_2\oplus Z_2\sim Z_4\)?}
  No.
  In \(Z_2\oplus Z_2\), every nonidentity element has order \(2\).
  But \(Z_4\) has an element of order \(4\) (namely \(\overline{1}\)).
  Since an isomorphism preserves element orders, \(Z_2\oplus Z_2\) cannot be isomorphic to \(Z_4\).
\end{solution}

\begin{exercise}
  If \(G\) is a group, then \(C=\{a\in G\mid ax=xa\text{ for all }x\in G\}\) is an abelian subgroup of \(G\).
  \(C\) is called the \textbf{center} of \(G\).
\end{exercise}

\begin{solution}
  Let
  \[
    C=\{a\in G \mid ax=xa \text{ for all } x\in G\}.
  \]

  \medskip\noindent\textbf{\(C\) is a subgroup of \(G\).}
  First, \(e\in C\) since \(ex=xe=x\) for all \(x\in G\); hence \(C\neq\varnothing\).
  Let \(a,b\in C\).
  For any \(x\in G\),
  \[
    (ab^{-1})x=a(b^{-1}x)=a(xb^{-1})=(ax)b^{-1}=(xa)b^{-1}=x(ab^{-1}),
  \]
  using that \(a\) and \(b\) commute with every element of \(G\).
  Thus \(ab^{-1}\in C\).
  By Theorem~2.5, \(C<G\).

  \medskip\noindent\textbf{\(C\) is abelian.}
  If \(a,b\in C\), then \(a\) commutes with every element of \(G\), in particular with \(b\); hence \(ab=ba\).
  Therefore \(C\) is abelian.

  Thus \(C\) is an abelian subgroup of \(G\), called the \emph{center} of \(G\).
\end{solution}

\begin{exercise}
  The group \(D_4^{\phantom{4}*}\) is not cyclic, but can be generated by two elements.
  The same is true of \(S_n\) (nontrivial).
  What is the minimal number of generators of the additive group \(\Z\oplus\Z\)?
\end{exercise}

\begin{solution}
  We claim that the additive group \(\mathbb{Z}\oplus\mathbb{Z}\) has minimal number of generators equal to \(2\).

  \medskip\noindent\textbf{(1) Two generators suffice.}
  Let \(e_1=(1,0)\) and \(e_2=(0,1)\).
  Then every \((m,n)\in\mathbb{Z}\oplus\mathbb{Z}\) can be written as
  \[
    (m,n)=m(1,0)+n(0,1)=m e_1+n e_2,
  \]
  so \(\mathbb{Z}\oplus\mathbb{Z}=\langle e_1,e_2\rangle\).

  \medskip\noindent\textbf{(2) One generator does not suffice.}
  If \(\mathbb{Z}\oplus\mathbb{Z}\) were generated by a single element \(v\), then it would be cyclic, i.e.\ \(\mathbb{Z}\oplus\mathbb{Z}=\langle v\rangle\).
  But any cyclic subgroup generated by \(v=(a,b)\) is
  \[
    \langle(a,b)\rangle=\{k(a,b):k\in\mathbb{Z}\},
  \]
  which lies on the line through the origin of slope \(b/a\) (or the \(y\)-axis if \(a=0\)).
  In particular, it cannot contain both \((1,0)\) and \((0,1)\).
  Hence \(\mathbb{Z}\oplus\mathbb{Z}\) is not cyclic.

  Therefore at least two generators are necessary.

  \medskip\noindent
  Combining (1) and (2), the minimal number of generators of \(\mathbb{Z}\oplus\mathbb{Z}\) is \(2\).
\end{solution}

\begin{exercise}
  If \(G=\langle a\rangle\) is a cyclic group and \(H\) is any group, then every homomorphism \(f:G\to H\) is completely determined by the element \(f(a)\in H\).
\end{exercise}

\begin{solution}
  Let \(G=\langle a\rangle\) be cyclic and let \(f:G\to H\) be a homomorphism.

  Every element of \(G\) has the form \(a^n\) for some \(n\in\Z\).
  Using the homomorphism property and induction on \(n\ge 0\), we have
  \[
    f(a^n)=f(a)^n \qquad (n\ge 0).
  \]
  For \(n<0\), write \(n=-m\) with \(m>0\). Then
  \[
    f(a^n)=f(a^{-m})=f(a^{-1})^m=f(a)^{-m}=f(a)^n,
  \]
  using \(f(a^{-1})=f(a)^{-1}\).
  Hence
  \[
    f(a^n)=f(a)^n \qquad \text{for all } n\in\Z.
  \]

  Therefore, for any \(g\in G\) with \(g=a^n\),
  \[
    f(g)=f(a^n)=f(a)^n,
  \]
  so the value of \(f\) on all of \(G\) is determined uniquely by the single element \(f(a)\in H\).
\end{solution}

\begin{exercise}
  The following cyclic subgroups are all isomorphic:
  the multiplicative group \(\langle i\rangle\) in \(\C\), the additive group \(Z_4\) and the subgroup \(\left\langle\begin{pmatrix}1&2&3&4\\2&3&4&1\end{pmatrix}\right\rangle\) of \(S_4\).
\end{exercise}

\begin{solution}
  Each of the three groups listed is cyclic of order \(4\), hence all three are isomorphic to \(Z_4\).
  We verify this explicitly.

  \medskip\noindent\textbf{(1) \(\langle i\rangle< \C^\times\).}

  Since \(i^4=1\) and the powers are
  \[
    i^0=1,\quad i^1=i,\quad i^2=-1,\quad i^3=-i,\quad i^4=1,
  \]
  the subgroup \(\langle i\rangle=\{1,i,-1,-i\}\) has \(4\) elements, so \(|\langle i\rangle|=4\).
  Thus \(\langle i\rangle\sim Z_4\) via
  \[
    \phi:Z_4\to \langle i\rangle,\qquad \phi(\overline{k})=i^k,
  \]
  which is a well-defined isomorphism (additive in \(Z_4\), multiplicative in \(\langle i\rangle\)).

  \medskip\noindent\textbf{(2) The subgroup generated by a 4-cycle in \(S_4\).}

  Let
  \[
    \sigma=\begin{pmatrix}1&2&3&4\\2&3&4&1\end{pmatrix}=(1\,2\,3\,4).
  \]
  Then
  \[
  \sigma^2=(1\,3)(2\,4),\qquad \sigma^3=(1\,4\,3\,2),\qquad \sigma^4=e,
  \]
 and \(\sigma^k\neq e\) for \(1\le k\le 3\).
 Hence \(|\langle\sigma\rangle|=4\), so \(\langle\sigma\rangle\sim Z_4\) via
  \[
    \psi:Z_4\to \langle\sigma\rangle,\qquad \psi(\overline{k})=\sigma^k,
  \]
  which is a well-defined isomorphism.

  \medskip\noindent\textbf{Conclusion.}
  
  Since \(\langle i\rangle\sim Z_4\) and \(\langle\sigma\rangle\sim Z_4\), it follows that all three cyclic groups are isomorphic.
\end{solution}

\begin{exercise}
  Let \(G\) be a group and \(\Aut G\) the set of all automorphisms of \(G\).
  \begin{enumerate}[label=(\alph*)]
    \item \(\Aut G\) is a group with composition of functions as binary operation.
    [\emph{Hint}: \(1_G\in\Aut G\) is an identity; inverses exist by Theorem 2.3.]
    \item \(\Aut\Z\sim Z_2\) and \(\Aut Z_6\sim Z_2\); \(\Aut Z_8\sim Z_2\oplus Z_2\); \(\Aut Z_p\sim Z_{p-1}\) (\(p\) prime).
    \item What is \(\Aut Z_n\) for arbitrary \(n\in\N^*\)?
  \end{enumerate}
\end{exercise}

\begin{solution}
  Let \(G\) be a group and \(\Aut(G)\) the set of all automorphisms of \(G\).

  \begin{enumerate}[label=(\alph*)]
    \item \textbf{\(\Aut(G)\) is a group under composition.}

    Composition of functions is associative.
    The identity map \(1_G:G\to G\) is an automorphism and serves as the identity element.
    If \(\varphi\in\Aut(G)\), then \(\varphi\) is bijective, so it has an inverse function \(\varphi^{-1}\); by Theorem~2.3 (inverse of an isomorphism is an isomorphism), \(\varphi^{-1}\) is also an automorphism.
    Hence every element has an inverse in \(\Aut(G)\), so \(\Aut(G)\) is a group.

    \item \textbf{Examples.}

    \smallskip\noindent\textbf{General fact for cyclic groups.}
    Let \(C=\langle a\rangle\) be cyclic of order \(n\).
    Any homomorphism \(f:C\to C\) is determined by \(f(a)=a^k\).
    Moreover, \(f\) is an automorphism iff \(f(a)\) is a generator of \(C\), i.e. iff \(\gcd(k,n)=1\).
    Thus
    \[
      \Aut(C)\ \sim\ (\Z/n\Z)^\times,
    \]
    via \(k\mapsto (a\mapsto a^k)\).

    Applying this:

    \begin{itemize}
      \item \( \Aut(\Z)\sim \{\pm 1\}\sim Z_2\), since an automorphism is determined by \(f(1)\in\Z\), and surjectivity forces \(f(1)=\pm 1\).

      \item \( \Aut(Z_6)\sim (\Z/6\Z)^\times=\{\overline{1},\overline{5}\}\sim Z_2.\)

      \item \( \Aut(Z_8)\sim (\Z/8\Z)^\times=\{\overline{1},\overline{3},\overline{5},\overline{7}\}\sim Z_2\oplus Z_2,\) since each nontrivial element has order \(2\) (e.g. \(\overline{3}^2=\overline{1}\), \(\overline{5}^2=\overline{1}\), \(\overline{7}^2=\overline{1}\)).

      \item If \(p\) is prime, then \( \Aut(Z_p)\sim (\Z/p\Z)^\times\), which is cyclic of order \(p-1\).
      Hence \(\Aut(Z_p)\sim Z_{p-1}\).
    \end{itemize}

    \item \textbf{\(\Aut(Z_n)\) for arbitrary \(n\in\mathbf{N}^*\).}

    Let \(Z_n=\langle \overline{1}\rangle\).
    Any homomorphism \(f:Z_n\to Z_n\) is determined by \(f(\overline{1})=\overline{k}\), and then \(f(\overline{m})=\overline{km}\).
    Such an \(f\) is an automorphism iff \(\overline{k}\) is a generator of the additive cyclic group \(Z_n\), i.e. iff \(\gcd(k,n)=1\).
    Therefore
    \[
      \Aut(Z_n)\ \sim\ (\Z/n\Z)^\times,
    \]
    the multiplicative group of units modulo \(n\).

    In particular,
    \[
      |\Aut(Z_n)|=\varphi(n),
    \]
    Euler's totient function.

    (\emph{Optional structure remark.}
    Using the Chinese remainder theorem and the decomposition \(n=\prod p_i^{e_i}\), one gets \((\Z/n\Z)^\times \sim \prod (\Z/p_i^{e_i}\Z)^\times\), so \(\Aut(Z_n)\) reduces to understanding prime powers.)
  \end{enumerate}
\end{solution}

\begin{exercise}
  For each prime \(p\) the additive subgroup \(Z(p^\infty)\) of \(\Q/\Z\) (Exercise 1.10) is generated by the set \(\{\overline{1/p^n}\mid n\in\N^*\}\).
\end{exercise}

\begin{solution}
  Fix a prime \(p\).
  Recall
  \[
    Z(p^\infty)=\{\overline{a/p^i}\in \Q/\Z\mid a\in\Z,\ i\ge 0\}.
  \]
  Let
  \[
    S=\left\{\overline{1/p^n}\ \middle|\ n\in\N^*\right\}\subset Z(p^\infty).
  \]
  We show that \(\langle S\rangle = Z(p^\infty)\).

  \medskip\noindent\textbf{(\(\subset\)).}
  Since \(S\subset Z(p^\infty)\) and \(Z(p^\infty)\) is a subgroup of \(\Q/\Z\), it follows that \(\langle S\rangle\subset Z(p^\infty)\).

  \medskip\noindent\textbf{(\(\supset\)).}
  Let \(\overline{a/p^i}\in Z(p^\infty)\) be arbitrary.
  If \(a\ge 0\), then in the additive group \(\Q/\Z\),
  \[
    \overline{\frac{a}{p^i}}=\underbrace{\overline{\frac{1}{p^i}}+\cdots+\overline{\frac{1}{p^i}}}_{a\ \text{times}} = a\,\overline{\frac{1}{p^i}}\in \langle S\rangle.
  \]
  If \(a<0\), write \(a=-m\) with \(m>0\).
  Then
  \[
    \overline{\frac{a}{p^i}}=-\,\overline{\frac{m}{p^i}}
  \]
  and \(\overline{m/p^i}\in\langle S\rangle\) by the previous case, hence \(\overline{a/p^i}\in\langle S\rangle\) as well (subgroups are closed under additive inverses).

  Thus every element of \(Z(p^\infty)\) lies in \(\langle S\rangle\), so \(Z(p^\infty)\subset \langle S\rangle\).

  \medskip\noindent
  Therefore \(\langle \overline{1/p^n}\mid n\in\N^*\rangle = Z(p^\infty)\).
\end{solution}

\begin{exercise}
  Let \(G\) be an abelian group and let \(H\), \(K\) be subgroups of \(G\).
  Show that the join \(H\vee K\) is the set \(\{ab\mid a\in H,b\in K\}\).
  Extend this result to any finite number of subgroups of \(G\).
\end{exercise}

\begin{solution}
  Let \(G\) be an abelian group and let \(H,K<G\).
  Recall that the join \(H\vee K\) is the subgroup of \(G\) generated by \(H\cup K\), i.e. \(H\vee K=\langle H\cup K\rangle\).

  Set
  \[
    S=\{ab\mid a\in H,\ b\in K\}.
  \]
  We show \(H\vee K=S\).

  \medskip\noindent\textbf{Step 1: \(S\) is a subgroup of \(G\).}
  Clearly \(e=e\cdot e\in S\), so \(S\neq\varnothing\).
  If \(a_1b_1,a_2b_2\in S\) with \(a_i\in H\), \(b_i\in K\), then (using that \(G\) is abelian)
  \[
    (a_1b_1)(a_2b_2)^{-1}=(a_1b_1)(b_2^{-1}a_2^{-1}) = a_1a_2^{-1}\, b_1b_2^{-1}\in H K,
  \]
  since \(a_1a_2^{-1}\in H\) and \(b_1b_2^{-1}\in K\).
  Hence \(S\) is closed under \(xy^{-1}\), so by Theorem~2.5 it is a subgroup of \(G\).

  \medskip\noindent\textbf{Step 2: \(H\vee K\subset S\).}
  Since \(H\subset S\) (take \(b=e\)) and \(K\subset S\) (take \(a=e\)), we have \(H\cup K\subset S\).
  Because \(S\) is a subgroup, it contains the subgroup generated by \(H\cup K\), i.e. \(H\vee K=\langle H\cup K\rangle\subset S\).

  \medskip\noindent\textbf{Step 3: \(S\subset H\vee K\).}
  If \(a\in H\) and \(b\in K\), then \(a\in H\vee K\) and \(b\in H\vee K\), hence \(ab\in H\vee K\).
  Therefore \(S\subset H\vee K\).

  Combining Steps 2 and 3 gives \(H\vee K=S=\{ab\mid a\in H,\ b\in K\}\).

  \medskip\noindent\textbf{Finite extension.}
  Let \(H_1,\dots,H_m<G\).
  Define
  \[
    S_m=\{a_1a_2\cdots a_m \mid a_i\in H_i\}.
  \]
  By the same argument (or by induction using the two-subgroup case), \(S_m\) is a subgroup of \(G\) containing each \(H_i\), hence it contains the join \(\bigvee_{i=1}^m H_i\).
  Conversely, every element of \(S_m\) is a product of elements from the \(H_i\), so it lies in the subgroup generated by \(\bigcup_i H_i\), i.e. in \(\bigvee_i H_i\).
  Thus
  \[
    \bigvee_{i=1}^m H_i \;=\; \{a_1a_2\cdots a_m \mid a_i\in H_i\}.
  \]
\end{solution}

\begin{exercise}
  \begin{enumerate}[label=(\alph*)]
    \item Let \(G\) be a group and \(\{H_i\mid i\in I\}\) a family of subgroups.
    State and prove a condition that will imply that \(\bigcup_{i\in I}H_i\) is a subgroup, that is, that \(\bigcup_{i\in I}H_i=\left\langle\bigcup_{i\in I}H_i\right\rangle\).
    \item Give an example of a group \(G\) and a family of subgroups \(\{H_i\mid i\in I\}\) such that \(\bigcup_{i\in I}H_i\ne\left\langle\bigcup_{i\in I}H_i\right\rangle\).
  \end{enumerate}
\end{exercise}

\begin{solution}
  \begin{enumerate}[label=(\alph*)]
    \item A sufficient (and standard) condition is that the family \(\{H_i\mid i\in I\}\) be \emph{linearly ordered by inclusion}:
    for all \(i,j\in I\), either \(H_i\subset H_j\) or \(H_j\subset H_i\).
    (That is, the family forms an ascending chain.)

    Assume this condition. Let
    \[
      H=\bigcup_{i\in I} H_i.
    \]
    We show that \(H<G\).
    Clearly \(H\neq\varnothing\) since each \(H_i\) is nonempty.
    Let \(a,b\in H\).
    Then \(a\in H_i\) and \(b\in H_j\) for some \(i,j\in I\).
    By the chain condition, we may assume \(H_i\subset H_j\) (after possibly interchanging \(i\) and \(j\)).
    Then \(a,b\in H_j\), so
    \[
      ab^{-1}\in H_j\subset H.
    \]
    Hence \(H\) is closed under \(ab^{-1}\).
    By Theorem~2.5, \(H\) is a subgroup of \(G\).
    In particular, \(H=\langle H\rangle=\left\langle\bigcup_{i\in I}H_i\right\rangle\).

    \item Example where the union is not a subgroup:
    take \(G=\Z\) (additively), \(H_1=2\Z\), \(H_2=3\Z\).
    Then
    \[
      H_1\cup H_2 = 2\Z\cup 3\Z
    \]
    is not a subgroup, since \(2\in H_1\) and \(3\in H_2\), but \(2+3=5\notin 2\Z\cup 3\Z\).
    On the other hand, \(\langle 2\Z\cup 3\Z\rangle=\Z\), because \(1=3-2\) lies in the subgroup generated by \(2\) and \(3\).
    Thus
    \[
      \bigcup_{i\in\{1,2\}} H_i \ne \left\langle\bigcup_{i\in\{1,2\}}H_i\right\rangle.
    \]
  \end{enumerate}
\end{solution}

\begin{exercise}
  \begin{enumerate}[label=(\alph*)]
    \item The set of all subgroups of a group \(G\), partially ordered by set theoretic inclusion, forms a complete lattice (Introduction, Exercises 7.1 and 7.2) in which the g.l.b. of \(\{H_i\mid i\in I\}\) is \(\bigcap_{i\in I}H_i\) and the l.u.b. is \(\left\langle\bigcup_{i\in I}H_i\right\rangle\).
    \item Exhibit the lattice of subgroups of the groups \(S_3\), \(D_4^{\phantom{4}*}\), \(Z_6\), \(Z_{27}\), and \(Z_{36}\).
  \end{enumerate}
\end{exercise}

\begin{solution}
  \begin{enumerate}[label=(\alph*)]
    \item Let \(\Sub(G)\) be the set of all subgroups of \(G\), ordered by inclusion.

    \smallskip
    \noindent\textbf{Greatest lower bounds.}
    If \(\{H_i\mid i\in I\}\subset \Sub(G)\), then \(\bigcap_{i\in I}H_i\) is a subgroup (nonempty since it contains \(e\), and closed under \(ab^{-1}\) because each \(H_i\) is).
    It is a lower bound, and if \(K\) is any subgroup with \(K\subset H_i\) for all \(i\), then \(K\subset \bigcap_i H_i\). Hence
    \[
      \glb\{H_i\}=\bigcap_{i\in I}H_i.
    \]

    \smallskip
    \noindent\textbf{Least upper bounds.}
    Let \(U=\bigcup_{i\in I}H_i\).
    Any upper bound \(K\) of the family satisfies \(H_i\subset K\) for all \(i\), hence \(U\subset K\).
    Since \(K\) is a subgroup containing \(U\), it contains the subgroup generated by \(U\), i.e. \(\langle U\rangle\subset K\).
    Thus \(\langle U\rangle\) is the least upper bound:
    \[
      \lub\{H_i\}=\left\langle\bigcup_{i\in I}H_i\right\rangle.
    \]
    Therefore \(\Sub(G)\) is a complete lattice with meet \(\wedge=\cap\) and join \(\vee=\langle\cup\rangle\).

    \item Below are the subgroup lattices (given as Hasse-diagram descriptions).
    Vertices are subgroups; an edge indicates \emph{covering} (no subgroup strictly in between).

    \medskip
    \noindent\textbf{(1) \(S_3\).}
    Subgroups:
    \[
      \{e\},\quad\langle(12)\rangle,\ \langle(13)\rangle,\ \langle(23)\rangle\ (\text{order }2),\quad A_3=\langle(123)\rangle\ (\text{order }3),\quad S_3.
    \]
    Inclusions (covers):
    \[
      \{e\}\lessdot \langle(12)\rangle,\ \langle(13)\rangle,\ \langle(23)\rangle,\ A_3, \qquad \langle(12)\rangle,\langle(13)\rangle,\langle(23)\rangle,\ A_3 \lessdot S_3.
    \]

    \medskip
    \noindent\textbf{(2) \(D_4^{\phantom{4}*}\) (order \(8\)).}
    Use the standard presentation \(D_4^{\phantom{4}*}=\langle r,s\mid r^4=e,\ s^2=e,\ srs=r^{-1}\rangle\).
    Subgroups (10 total):
    \[
      \{e\},\ \langle r^2\rangle;
    \]
    four reflection subgroups of order \(2\):
    \[
      \langle s\rangle,\ \langle sr\rangle,\ \langle sr^2\rangle,\ \langle sr^3\rangle;
    \]
    one cyclic subgroup of order \(4\):
    \[
      \langle r\rangle=\{e,r,r^2,r^3\};
    \]
    two Klein-four subgroups:
    \[
      V_1=\langle r^2,s\rangle=\{e,r^2,s,sr^2\},\qquad V_2=\langle r^2,sr\rangle=\{e,r^2,sr,sr^3\};
    \]
    and \(D_4^{\phantom{4}*}\) itself.

    Cover relations:
    \[
      \{e\}\lessdot \langle r^2\rangle,\ \langle s\rangle,\ \langle sr\rangle,\ \langle sr^2\rangle,\ \langle sr^3\rangle;
    \]
    \[
      \langle r^2\rangle \lessdot \langle r\rangle,\ V_1,\ V_2;
    \]
    \[
      \langle s\rangle,\langle sr^2\rangle \lessdot V_1,\qquad \langle sr\rangle,\langle sr^3\rangle \lessdot V_2;
    \]
    \[
      \langle r\rangle,\ V_1,\ V_2 \lessdot D_4^{\phantom{4}*}.
    \]

    \medskip
    \noindent\textbf{(3) \(Z_6\) (additive).}
    Since \(Z_6\) is cyclic, there is exactly one subgroup for each divisor of \(6\):
    orders \(1,2,3,6\).
    Concretely:
    \[
      \{0\},\quad \langle 3\rangle\ (\text{order }2),\quad \langle 2\rangle\ (\text{order }3),\quad Z_6.
    \]
    This lattice has the two chains:
    \[
      \{0\}\lessdot \langle 3\rangle \lessdot Z_6 \quad\text{and}\quad \{0\}\lessdot \langle 2\rangle \lessdot Z_6,
    \]
    with \(\langle 2\rangle\) and \(\langle 3\rangle\) incomparable.

    \medskip
    \noindent\textbf{(4) \(Z_{27}\).}
    Divisors are \(1,3,9,27\), hence unique subgroups of these orders:
    \[
      \{0\}\lessdot \langle 9\rangle \lessdot \langle 3\rangle \lessdot Z_{27}.
    \]
    (Here \(\langle 3\rangle\) has order \(9\), \(\langle 9\rangle\) has order \(3\).)

    \medskip
    \noindent\textbf{(5) \(Z_{36}\).}
    Divisors of \(36\) are \(1,2,3,4,6,9,12,18,36\), hence one subgroup of each order.
    A convenient label is \(H_d\) for the unique subgroup of order \(d\).
    The cover relations (Hasse edges) correspond to \emph{maximal} proper inclusions, i.e. \(H_{d_1}\lessdot H_{d_2}\) when \(d_1\mid d_2\) and there is no divisor strictly between them.

    Covers are:
    \[
      H_1 \lessdot H_2,\ H_3;
    \]
    \[
      H_2 \lessdot H_4,\ H_6;\qquad H_3 \lessdot H_6,\ H_9;
    \]
    \[
      H_4 \lessdot H_{12};\qquad H_6 \lessdot H_{12},\ H_{18};\qquad H_9 \lessdot H_{18};
    \]
    \[
      H_{12}\lessdot H_{36};\qquad H_{18}\lessdot H_{36}.
    \]
    (Equivalently, you can picture this as the divisor lattice of \(36\), turned upside down.)
  \end{enumerate}
\end{solution}
