\section{The Axiom of Choice, Order, and Zorn's Lemma}

\begin{exercise}
  Let $(A, \leq)$ be a partially ordered set and $B$ a nonempty subset.
  A \textbf{lower bound} of $B$ is an element $d \in A$ such that $d \leq b$ for \emph{every} $b \in B$.
  A \textbf{greatest lower bound (g.l.b.)} of $B$ is a lower bound $d_0$ of $B$ such that $d \leq d_0$ for every other lower bound $d$ of $B$.
  A \textbf{least upper bound (l.u.b.)} of $B$ is an upper bound $t_0$ of $B$ such that $t_0 \leq t$ for every other upper bound $t$ of $B$.
  $(A, \leq)$ is a \textbf{lattice} if for all $a, b \in A$ the set $\{a, b\}$ has both a greatest lower bound and a least upper bound.
  \begin{enumerate}[label=(\alph*)]
    \item If $S \neq \varnothing$, then the power set $P(S)$ ordered by set-theoretic inclusion is a lattice, which has a unique maximal element.
    \item Give an example of a partially ordered set which is \emph{not} a lattice.
    \item Give an example of a lattice with no maximal element and an example of a partially ordered set with two maximal elements.
  \end{enumerate}
\end{exercise}

\begin{solution}
  \begin{enumerate}[label=(\alph*)]
    \item For $X, Y \subset S$ the greatest lower bound is
    \[
      X\cap Y.
    \]
    The least upper bound is
    \[
      X\cup Y.
    \]
    Thus every pair $X, Y$ has a g.l.b. and l.u.b., so $(P(S), \subset)$ is a lattice.

    A maximal element in $P(S)$ is an element that is not properly contained in any other element.
    The whole set $S$ is an upper bound for every subset of $S$ and is not contained in any strictly larger subset of $S$, so $S$ is a maximal element.
    It is unique because if $T$ is any subset with $U \subset T$ for all $U \subset S$, then in particular $S \subset T$, so $T = S$.

    \item Take the set $A = \{a, b\}$ with the only order relations being reflexivity:
    \[
      a\leq a,\qquad b\leq b,
    \]
    For the pair $a, b$ there is no lower bound other than possibly elements $\leq a$ and $\leq b$; but the only candidates are $a$ and  $b$ themselves, and neither is $\leq$ the other.
    Hence there is no greatest lower bound of $a, b$.
    (Similarly there is no least upper bound.) Therefore this poset is not a lattice.

    \item Take the integers $\Z$ with the usual order.
    For any $m, n \in \Z$ the least upper bound is $\max m, n$ and the greatest lower bound is $\min m, n$; thus $(\Z, \leq)$ is a lattice.
    But $\Z$ has no maximal element because for every $n \in \Z$ there exists $n + 1 > n$.
    So $\Z$ is a lattice with no maximal element.
    
    Let $A = \{0, a, b\}$ and define the order by
    \[
      0\le a,\qquad 0\le b.
    \]
  \end{enumerate}
\end{solution}

\begin{exercise}
  A lattice $(A, \leq)$ (see Exercise 1) is said to be \textbf{complete} if every nonempty subset of $A$ has both a least upper bound and a greatest lower bound.
  A map of partially ordered sets $f : A \to B$ is said to preserve order if $a \leq a'$ in $A$ implies $f(a) \leq f(a')$ in $B$.
  Prove that an order-preserving map $f$ of a complete lattice $A$ into itself has at least one fixed element (that is, an $a \in A$ such that $f(a) = a$).
\end{exercise}

\begin{solution}
  Let $S = \{\,a \in A : f(a) \leq a\,\}$ be the set of all pre-fixed points of $f$.  
  Since $A$ is complete, it has a greatest element, say $1$.  
  Because $f$ preserves order, $f(1) \leq 1$, so $1 \in S$.  
  Thus $S \neq \varnothing$ and, since $A$ is complete, $S$ has a g.l.b; call it
  \[
    m = \inf S.
  \]

  \textit{First}, we show that $f(m) \leq m$.  
  For every $s \in S$ we have $m \leq s$, hence $f(m) \leq f(s)$ by order preservation.  
  Since $s \in S$, $f(s) \leq s$, and thus $f(m) \leq s$ for all $s \in S$.  
  Hence $f(m)$ is a lower bound of $S$, and by maximality of $m$ as greatest lower bound, $f(m) \leq m$.

  \textit{Second}, we show that $m \leq f(m)$.  
  Since $m$ is a lower bound of $S$ and $f$ is order-preserving, the argument above shows that $f(m)$ is also a lower bound of $S$.  
  Therefore $f(m) \leq s$ for all $s \in S$, so $f(m)$ is a lower bound of $S$.  
  Because $m$ is the greatest lower bound, we must have $m \leq f(m)$.

  Combining the inequalities $f(m) \leq m$ and $m \leq f(m)$, we conclude that $f(m) = m$.  
  Thus $f$ has a fixed element.
\end{solution}

\begin{exercise}
  Exhibit a well ordering of the set $\Q$ of rational numbers.
\end{exercise}

\begin{solution}
  Write each rational number in \(\Q\) in its unique reduced form \(a/b\) with \(b>0\) and \(\gcd(a,b)=1\).
  (Under this convention the rational \(0\) is represented uniquely as \(0/1\).)

  Define a binary relation \(\trianglelefteq\) on \(\Q\) by declaring
  \[
    \frac{a}{b}\trianglelefteq\frac{c}{d}
  \]
  iff either
  \begin{enumerate}
    \item \( |a|+b<|c|+d\), or
    \item \( |a|+b=|c|+d\) and \(a<c\), or
    \item \( |a|+b=|c|+d,\ a=c\), and \(b\le d\).
  \end{enumerate}
  Since every rational is written in the unique reduced form specified above, the quantities \(|a|+b\), \(a\), and \(b\) are well defined for each rational, so \(\trianglelefteq\) is well defined.

  It is immediate that \(\trianglelefteq\) is a total order.
  To see that it is a well ordering, let \(S\subseteq\Q\) be nonempty and for each \(x=a/b\in S\) set \(N(x)=|a|+b\in\N\).
  The set \(\{N(x):x\in S\}\) is a nonempty subset of \(\N\), hence has a least element \(n_0\).
  The subset \(T=\{x\in S:N(x)=n_0\}\) is therefore nonempty.
  Among elements of \(T\), the numerators form a finite (hence well-ordered) subset of \(\Z\), so there is a least numerator \(a_0\).
  Finally, among rationals in \(T\) with numerator \(a_0\) the denominator is minimal for the \(\trianglelefteq\)-least element.
  Thus \(T\) (and hence \(S\)) has a least element with respect to \(\trianglelefteq\).
  Therefore \(\trianglelefteq\) is a well ordering of \(\Q\).
\end{solution}

\begin{exercise}
  Let $S$ be a set.
  A \textbf{choice function} for $S$ is a function $f$ from the set of all nonempty subsets of $S$ to $S$ such that $f(A) \in A$ for all $A \neq \varnothing$, $A \subset S$.
  Show that the Axiom of Choice is equivalent to the statement that every set $S$ has a choice function.
\end{exercise}

\begin{solution}
  We show the two statements are equivalent.

  \medskip\noindent\textbf{(AC \(\Rightarrow\) choice functions exist).}
  Let \(S\) be any set and let \(\mathcal{I}\) denote the collection of all nonempty subsets of \(S\).
  If \(\mathcal{I}=\varnothing\) then \(S=\varnothing\), and the unique function \(\varnothing\to\varnothing\) is a choice function for \(S\).
  Thus assume \(\mathcal{I}\neq\varnothing\).
  Consider the family \(\{X_A\}_{A\in\mathcal{I}}\) where \(X_A=A\) for each \(A\in\mathcal{I}\).
  Every \(X_A\) is nonempty by definition, and the family is indexed by the nonempty set \(\mathcal{I}\).
  By the Axiom of Choice (the product of a family of nonempty sets indexed by a nonempty set is nonempty), the product \(\prod_{A\in\mathcal{I}} X_A\) is nonempty.
  An element of this product is precisely a function \(f\colon\mathcal{I}\to S\) with \(f(A)\in X_A=A\) for each \(A\); that is exactly a choice function for \(S\).
  Hence every set \(S\) admits a choice function.

  \medskip\noindent\textbf{(Choice functions exist \(\Rightarrow\) AC).}
  Assume every set \(T\) admits a choice function \(c_T\) defined on the collection of nonempty subsets of \(T\).
  Let \(\{X_i\}_{i\in I}\) be any family of nonempty sets indexed by a nonempty set \(I\).
  Put \(S=\bigcup_{i\in I} X_i\).
  Then each \(X_i\) is a nonempty subset of \(S\), so the hypothesis supplies a choice function \(c_S\) for \(S\).
  Define \(g\colon I\to S\) by \(g(i):=c_S(X_i)\). By construction \(g(i)\in X_i\) for every \(i\in I\), so \(g\in\prod_{i\in I}X_i\).
  Hence the product is nonempty.
  This establishes the Axiom of Choice.

  \medskip\noindent Therefore the two statements are equivalent.
\end{solution}

\begin{exercise}
  Let $S$ be the set of all points $(x, y)$ in the plane with $y \leq 0$.
  Define an ordering by $(x_1, y_1) \leq (x_2, y_2) \iff x_1 = x_2 \text{ and } y_1 \leq y_2$.
  Show that this is a partial ordering of $S$, and that $S$ has infinitely many maximal elements.
\end{exercise}

\begin{solution}
  Let \(S=\{(x,y)\in\R^2: y\leq 0\}\) and define
  \[
  (x_1,y_1)\leq (x_2,y_2)\iff x_1=x_2\ \text{and}\ y_1\leq y_2 .
  \]

  \textbf{(i) This relation is a partial order.}
  \begin{itemize}
    \item \emph{Reflexive:} For any \((x,y)\in S\) we have \(x=x\) and \(y\leq y\), so \((x,y)\leq (x,y)\).
    \item \emph{Antisymmetric:} If \((x_1,y_1)\leq (x_2,y_2)\) and \((x_2,y_2)\leq (x_1,y_1)\), then \(x_1=x_2\) and \(y_1\leq y_2\), and also \(x_2=x_1\) and \(y_2\leq y_1\).
    Hence \(y_1=y_2\) and therefore \((x_1,y_1)=(x_2,y_2)\).
    \item \emph{Transitive:} If \((x_1,y_1)\leq (x_2,y_2)\) and \((x_2,y_2)\leq (x_3,y_3)\), then \(x_1=x_2\) and \(x_2=x_3\), so \(x_1=x_3\), and \(y_1\leq y_2\leq y_3\), hence \(y_1\leq y_3\).
    Thus \((x_1,y_1)\leq (x_3,y_3)\).
  \end{itemize}
  Therefore the relation is reflexive, antisymmetric, and transitive, i.e. a partial order.

  \bigskip

  \textbf{(ii) \(S\) has infinitely many maximal elements.}

  Fix any real number \(x_0\).
  For that \(x_0\) the point \((x_0,0)\in S\) satisfies the following: if \((x_0,0)\leq  (x,y)\) then \(x=x_0\) and \(0\leq y\).
  Since every element of \(S\) has \(y\leq 0\), the only possibility is \(y=0\), so \((x,y)=(x_0,0)\).
  Thus there is no element of \(S\) strictly greater than \((x_0,0)\); i.e. \((x_0,0)\) is maximal.

  As \(x_0\) ranges over \(\R\) we obtain the family \(\{(x,0):x\in\R\}\) of maximal elements,which is infinite (indeed uncountable).
  Hence \(S\) has infinitely many maximal elements.

  (Observe also that any point \((x,y)\) with \(y<0\) is not maximal because \((x,y)<(x,0)\).)
\end{solution}

\begin{exercise}
  Prove that if all the sets in the family $\{A_i \mid i \in I \neq \varnothing\}$ are nonempty, then each of the projections $\pi_k \colon \prod_{i \in I} A_i \to A_k$ is surjective.
\end{exercise}

\begin{solution}
  Let \(\{A_i\}_{i\in I}\) be a family of sets with \(A_i\neq\varnothing\) for each \(i\in I\).
  Fix \(k\in I\) and let \(\pi_k:\prod_{i\in I}A_i\to A_k\) be the projection onto the \(k\)-th coordinate.
  We must show that \(\pi_k\) is surjective, i.e. that for every \(a\in A_k\) there exists \(f\in\prod_{i\in I}A_i\) with \(\pi_k(f)=f(k)=a\).

  For a given \(a\in A_k\) we need to define a function \(f:I\to\bigcup_{i\in I}A_i\) such that \(f(i)\in A_i\) for all \(i\in I\) and \(f(k)=a\).
  To do this we must choose, for each \(i\in I-\{k\}\), an element \(f(i)\in A_i\).
  The existence of a choice function selecting one element from each \(A_i\) (for \(i\neq k\)) is exactly an instance of the Axiom of Choice.
  Assuming Choice (or equivalently the hypothesis that the product \(\prod_{i\in I}A_i\) is nonempty), pick such elements \(f(i)\) for all \(i\neq k\), and put \(f(k)=a\).
  Then \(f\in\prod_{i\in I}A_i\) and \(\pi_k(f)=a\).
  Since \(a\) was arbitrary, \(\pi_k\) is surjective.

  \medskip

  \noindent\textbf{Remark.} If the index set \(I\) is finite, no form of the Axiom of Choice is needed: one can choose elements from the finitely many \(A_i\) inductively (or by a finite product of nonempty sets being nonempty).
  The use of Choice becomes essential only when \(I\) is infinite.
\end{solution}

\begin{exercise}
  Let $(A, \leq)$ be a linearly ordered set.
  The \textbf{immediate successor} of $a \in A$ (if it exists) is the least element in the set $\{x \in A \mid a < x\}$.
  Prove that if $A$ is well ordered by $\leq$, then at most one element of $A$ has no immediate successor.
  Give an example of a linearly ordered set in which precisely two elements have no immediate successor.
\end{exercise}

\begin{solution}
  First remark: if \(a\in A\) has no immediate successor, that means the set \(\{x\in A:x>a\}\) either is empty (so \(a\) is maximal) or is nonempty but has no least element.

  \medskip\noindent\textbf{At most one element has no immediate successor.}
  Suppose for contradiction that \(a\) and \(b\) are two distinct elements of \(A\) with no immediate successor.
  Since \(A\) is linearly ordered, either \(a<b\) or \(b<a\).
  Without loss of generality assume \(a<b\).
  Then \(b\in\{x\in A:x>a\}\), so this set is nonempty.
  But \(A\) is well ordered, hence every nonempty subset has a least element; therefore \(\{x\in A:x>a\}\) has a least element \(c\).
  By definition \(c\) is the immediate successor of \(a\), contradicting the assumption that \(a\) has no immediate successor.
  Thus it is impossible for two distinct elements to both lack immediate successors; at most one element of \(A\) can have no immediate successor.
  \(\square\)

  \medskip\noindent\textbf{Example with exactly two elements having no immediate successor.}
  Let
  \[
    B=\{0\}\cup\{1/n : n\in\mathbf{N}^*\}\subset\R
  \]
  equipped with the usual order inherited from \(\R\).
  Every element of \(B\) except \(0\) is of the form \(1/n\) for some \(n\in\mathbf{N}^*\).
  For \(n\geq 2\), the least element strictly greater than \(1/n\) is \(1/(n-1)\), so \(1/n\) has an immediate successor.
  The element \(1=1/1\) is maximal in \(B\) (no larger element of \(B\) exists), hence it has no immediate successor.
  The element \(0\) also has no immediate successor: the set \(\{x\in B:x>0\}=\{1/n:n\in\mathbf{N}^*\}\) has no least element because for each \(1/n\) there is \(1/(n+1)\in B\) with \(0<1/(n+1)<1/n\).
  Therefore \(0\) has no immediate successor. No other elements of \(B\) lack immediate successors, so exactly two elements of \(B\) (namely \(0\) and \(1\)) have no immediate successor.
\end{solution}
